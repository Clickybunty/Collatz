\documentclass[a4paper,12pt]{article}
\usepackage{amsmath}
\usepackage{amssymb}
\usepackage{hyperref}
\usepackage[utf8]{inputenc}
\usepackage[ngerman]{babel}
\usepackage{geometry}
\usepackage{fancyhdr}
\pagestyle{fancy}
\fancyhead{} % Remove header content
\renewcommand{\headrulewidth}{0pt} % Remove the top line
\fancyfoot[C]{Version 1.1 – Correction: Logarithmic instead of exponential reduction}
\fancyfoot[R]{\thepage} % Page number on the right

\geometry{a4paper, margin=1in}

\title{Collatz Conjecture – Explanation of the Central Dynamics and Convergence}

\author{Stevan Menicanin}
\date{\today}

\begin{document}

\maketitle

\renewcommand{\abstractname}{Abstract}
\begin{abstract}
    The Collatz Conjecture, also known as the (3n+1) problem, is one of the most fascinating unsolved problems in mathematics. This paper examines the central mechanics of the Collatz transformation and demonstrates how the \(+1\) operator, the logarithmic bound, and the structure of the transformation are utilized to explain universal convergence.

    A general bound is developed that guarantees every natural number \( n \) is reduced after a finite number of steps. Furthermore, a precise congruence analysis proves that no new stable cycles exist outside the known cycle \( \{4, 2, 1\} \). The \(+1\) operator plays a key role in creating asymmetry and enabling a logarithmic reduction.


    The presented results provide strong evidence supporting the validity of the Collatz Conjecture and reveal the underlying mathematical order. This paper is intended as a contribution to the mathematical discussion and requires independent verification by the scientific community.
\end{abstract}

\newpage
\setcounter{tocdepth}{2} % Legt die Tiefe des Inhaltsverzeichnisses fest (z. B. 2 für subsections)

\tableofcontents

\newpage

\section{Introduction}
The \textbf{Collatz Conjecture}, also known as the (3n+1) problem, was first formulated in the 1930s by Lothar Collatz. It states:

\begin{quote}
\textit{Start with any positive integer \( n \). Repeatedly apply the following transformation:}
\begin{itemize}
    \item \textbf{If \( n \) is even:} Divide it by 2: \( T(n) = \frac{n}{2} \).
    \item \textbf{If \( n \) is odd:} Multiply it by 3 and add 1: \( T(n) = 3n + 1 \).
\end{itemize}
\textit{The transformation is repeated until \( n = 1 \) is reached, assuming \( n \) does not remain odd indefinitely. The conjecture states that every positive integer \( n \) will, after a finite number of steps, reach the cycle \( 4 \to 2 \to 1 \to 4 \).}
\end{quote}

Despite numerous numerical confirmations, a general proof remains elusive to this day. This paper aims to prove the Collatz Conjecture through a systematic analysis.
\section{Relation to Previous Research}
The Collatz Conjecture has already been investigated through numerous numerical tests, extending to extremely large starting values. Works such as those by Jeffrey C. Lagarias analyze the dynamic properties of the transformation.\footnote{J. C. Lagarias, "The 3x+1 Problem and its Generalizations", \textit{American Mathematical Monthly}, vol. 92, no. 1, pp. 3–23, 1985.} This paper expands the discussion by providing a systematic explanation of the dynamics through the logarithmic bound and the role of the \(+1\) operator.

The methods presented build upon the numerical validation of the conjecture and offer a theoretical perspective to explain the universal convergence of the Collatz transformation.

\section{Mathematical Foundation of the Transformation}
The transformation \( T(n) \) is based on two fundamental cases:
\begin{enumerate}
    \item For even numbers \( n \), \( T(n) = \frac{n}{2} \).
    \item For odd numbers \( n \), \( T(n) = 3n + 1 \), followed by repeated divisions by 2.
\end{enumerate}

The deterministic nature of the transformation ensures that every starting number \( n \) generates a uniquely defined sequence. The seemingly chaotic behavior arises from alternating phases of growth and logarithmic reduction, structured by the \(+1\) operator.

\section{The \(+1\) Operator and Asymmetry}
The \(+1\) operator is critical to the dynamics of the Collatz Conjecture:
\begin{itemize}
    \item It ensures that odd numbers are transformed into even numbers.
    \item It creates an asymmetry that prevents trivial or periodic cycles outside of \( \{4, 2, 1\} \) from forming.
\end{itemize}

\subsection{Mathematical Analysis of the \(+1\) Operator}
For odd numbers \( n \), the following holds:
\[
T(n) = 3n + 1.
\]
The result is always an even number because:
\[
3n \mod 2 = 1 \quad \text{and therefore} \quad (3n + 1) \mod 2 = 0.
\]

The asymmetry arises because the \(+1\) operator prevents odd numbers from being directly transformed back into other odd numbers. This enables logarithmic reduction through repeated divisions.

\subsection{Long-Term Influence of the \(+1\) Operator}
Although the relative influence of the \(+1\) operator diminishes for large numbers:
\[
\lim_{n \to \infty} \frac{+1}{3n} = 0,
\]
its role remains critical because it:
\begin{itemize}
    \item ensures divisibility after each \( 3n+1 \) operation,
    \item permanently maintains asymmetry, and
    \item eliminates any possibility of new cycles.
\end{itemize}

\section{The Logarithmic Bound}
The logarithmic bound states that for every number \( n \in \mathbb{N} \), there exists a \( k \in \mathbb{N} \) such that:
\[
T_k(n) = \frac{3n + 1}{2^k} < n.
\]
This bound shows that the number of steps required for complete reduction grows asymptotically logarithmically:
\[
O(\log_2 n).
\]
This means that larger numbers require more steps, but the growth in the number of steps increases only slowly.  
This guarantees that every number is reduced after a finite number of steps, following a logarithmic rather than an exponential pattern.


\subsection{Proof of the Bound}
For odd numbers, the following holds:
\[
T_k(n) = \frac{3n + 1}{2^k}.
\]
The condition \( T_k(n) < n \) leads to:
\[
2^k > 3 + \frac{1}{n}.
\]
By taking the logarithm, we obtain:
\[
k > \log_2(3 + \frac{1}{n}).
\]
For large \( n \), \( \frac{1}{n} \to 0 \), so:
\[
k > \log_2(3) \approx 1.585.
\]

\subsection{Cumulative Probabilities of Reduction}
The probability that a number is reduced after \( k \) steps is:
\[
P_k = \frac{1}{2^k}.
\]
The cumulative probability that a number is reduced after \( k \) steps is given as the sum:
\[
P_{\text{cumulative}} = \sum_{i=1}^k \frac{1}{2^i}.
\]
In the limit, this sum converges to 1:
\[
\lim_{k \to \infty} P_{\text{cumulative}} = 1.
\]
This demonstrates that the probability of long-term reduction approaches 1, effectively limiting growth.

\section{Numerical Validation and Simulations}
To support the theoretical argument, numerical simulations were conducted. Some well-known starting values illustrate the practical effectiveness of the logarithmic bound:

\begin{itemize}
    \item For \( n = 27 \): The transformation reaches the cycle \( \{4, 2, 1\} \) after 111 steps. It becomes evident that growth phases (e.g., \( T(27) = 82 \), \( T(82) = 41 \)) are offset by logarithmic reductions.
    \item For \( n = 7 \): After 16 steps, \( T(n) \) reaches the cycle \( \{4, 2, 1\} \). This illustrates the asymmetry and the role of the \(+1\) operator in the transformation.
\end{itemize}

These simulations highlight that the theoretical bounds are valid even for practical ranges of numbers.

\section{Ensuring the Sufficient Number of Steps \( k \)}

\subsection{Analysis of the Bound and Extension of the Proof}
The central task is to ensure that the number of steps \( k \) in the transformation \( T_k(n) \) is always large enough to guarantee \( T_k(n) < n \). To this end, we consider the logarithmic bound:
\[
k > \log_2\left(3 + \frac{1}{n}\right).
\]
For \( n \to \infty \), the right-hand expression converges to \( \log_2(3) \approx 1.585 \). Since \( k \) is a natural number, it must hold that \( k \geq 2 \). We demonstrate that \( k \) is sufficiently large in all cases to satisfy this condition.

\subsection{Extension of the Induction Proof}
\subsubsection{Base Case (Extended)}
For \( n = 1 \), the following holds:
\[
T(1) = 4, \quad T(4) = 2, \quad T(2) = 1.
\]
After exactly \( k = 3 \) steps, the transformation reaches the cycle \( \{4, 2, 1\} \). Thus:
\[
T_3(1) = 1 < 4.
\]
Therefore, the base case is satisfied for \( k = 3 \).

\subsubsection{Induction Hypothesis (Extended)}
Assume that for all \( n \leq m \), there exists a \( k \) such that \( T_k(n) < n \) holds. Furthermore, assume \( k \) is always sufficiently large to satisfy the bound:
\[
k > \log_2\left(3 + \frac{1}{n}\right).
\]

\subsubsection{Induction Step (Extended)}
We show that the statement also holds for \( n = m + 1 \).

\paragraph{Case \( m + 1 \) is even:}
In this case:
\[
T(m + 1) = \frac{m + 1}{2}.
\]
The number of steps \( k \) is given by:
\[
k \geq \log_2(m + 1).
\]
Since \( \log_2(m + 1) \) for all \( m + 1 \geq 2 \) is greater than or equal to the bound \( \log_2(3) \), \( k \) is always sufficiently large to ensure \( T_k(m + 1) < m + 1 \).

\paragraph{Case \( m + 1 \) is odd:}
In this case:
\[
T(m + 1) = 3(m + 1) + 1.
\]
The result is an even number, which is exponentially reduced through repeated division by 2:
\[
T_k(m + 1) = \frac{3(m + 1) + 1}{2^k}.
\]
For \( k \geq \log_2\left(3 + \frac{1}{m + 1}\right) \), the condition
\[
T_k(m + 1) < m + 1
\]
is always satisfied. The bound \( k > \log_2\left(3 + \frac{1}{n}\right) \) ensures that \( k \) cannot be too small.

\subsubsection{Conclusion}
Since both the base case and the induction step have been shown, it follows by complete induction that \( k \) is sufficiently large for all \( n \in \mathbb{N} \) to satisfy the bound:
\[
T_k(n) < n.
\]

\subsection{Conclusion on the Sufficient Number of Steps}
The logarithmic bound and the extended induction argument demonstrate that \( k \) can never be too small to ensure \( T_k(n) < n \). The key is the logarithmic reduction by \( 2^k \), which guarantees that \( T_k(n) \) becomes smaller than \( n \) after a finite number of steps for every natural number \( n \).

\section{Inductive Proof of the Logarithmic Bound}
We prove that for all \( n \in \mathbb{N} \):
\[
\exists k \in \mathbb{N}, \text{ such that } T_k(n) < n.
\]

\subsection{Base Case}
For \( n = 1 \), the following holds:
\[
T(1) = 4, \quad T(4) = 2, \quad T(2) = 1.
\]
Thus, the statement is satisfied for \( n = 1 \).

\subsection{Induction Hypothesis}
Assume that the statement holds for all numbers \( n \leq m \), i.e., there exists a \( k \) such that \( T_k(n) < n \).

\subsection{Induction Step}
We show that the statement also holds for \( n = m + 1 \):
\begin{itemize}
    \item If \( m + 1 \) is even, then:
    \[
    T(m + 1) = \frac{m + 1}{2}.
    \]
    After \( \log_2(m + 1) \) steps, \( T(m + 1) < m + 1 \), which satisfies the statement.
    \item If \( m + 1 \) is odd, then:
    \[
    T(m + 1) = 3(m + 1) + 1.
    \]
    The result is an even number, which is exponentially reduced through repeated division by 2. Based on the above analysis, there exists a \( k \) such that:
    \[
    T_k(m + 1) < m + 1.
    \]
\end{itemize}

Thus, the statement is proven for \( n = m + 1 \). By complete induction, the logarithmic bound holds for all \( n \in \mathbb{N} \).

\section{Extended Analysis: Exclusion of New Cycles and the Role of the \(+1\) Operator}

\subsection{Congruence Analysis: Exclusion of New Cycles}
A new cycle \( C = \{n_1, n_2, \dots, n_k\} \) would need to satisfy the following conditions:
\begin{itemize}
    \item After \( k \) transformations, \( T^k(n) = n \) for all \( n \in C \).
    \item The sequence must not intersect the known cycle \( \{4, 2, 1\} \).
\end{itemize}

Consider the congruence \( 3n + 1 \equiv n \pmod{2^m} \). This can be rearranged as:
\[
2n + 1 \equiv 0 \pmod{2^m}.
\]
This implies that \( 2n + 1 \) must be an odd number divisible by \( 2^m \). However, for positive integers \( n \), this is only possible under very specific conditions:
\begin{enumerate}
    \item \( n \) must have the structure \( 2n + 1 = k \cdot 2^m \), where \( k \) is an odd integer.
    \item This can only hold for \( n \in \{4, 2, 1\} \), as all other \( n \) are either exponentially reduced or transformed into the known cycle.
\end{enumerate}

An example demonstrates this:
\begin{itemize}
    \item For \( n = 5 \): 
    \[
    T(5) = 3 \cdot 5 + 1 = 16, \quad T(16) = 8, \quad T(8) = 4.
    \]
    The sequence inevitably enters the known cycle \( \{4, 2, 1\} \).
    \item For \( n = 7 \): 
    \[
    T(7) = 3 \cdot 7 + 1 = 22, \quad T(22) = 11, \quad T(11) = 34, \quad \dots
    \]
    Again, the transformation ultimately leads to the known cycle.
\end{itemize}

It follows that any number not in the cycle \( \{4, 2, 1\} \) is either reduced or transformed into this cycle. Therefore, new stable cycles are excluded.

\subsection{The Role of the \(+1\) Operator}
The \(+1\) operator in the transformation \( T(n) = 3n + 1 \) plays a crucial role:
\begin{itemize}
    \item It creates the necessary asymmetry that prevents odd numbers from being directly transformed back into other odd numbers.
    \item For small numbers, the \(+1\) operator has a significant impact, ensuring that every odd number is transformed into an even number.
\end{itemize}

Mathematically expressed:
\[
3n + 1 \quad \text{is always even because } \quad 3n \mod 2 = 1.
\]

\subsubsection{Diminishing Influence for Large Numbers}
For large values of \( n \), the influence of the \(+1\) operator becomes negligibly small:
\[
\lim_{n \to \infty} \frac{+1}{3n} = 0.
\]
Nevertheless, its role remains indispensable, as it ensures the transition to logarithmic reduction through divisions by \( 2^k \). Without the \(+1\) operator, growth under \( 3n \) could become uncontrolled.

\subsection{Prevention of Infinite Growth}
Without the \(+1\) operator, \( T(n) = 3n \) for odd \( n \) could theoretically grow indefinitely. However, the \(+1\) operator ensures that every odd number is transformed into an even number, which is then exponentially reduced:
\[
T_k(n) = \frac{3n + 1}{2^k} \quad \text{leads to } \quad T_k(n) < n \quad \text{after a finite number of steps}.
\]
This demonstrates that the \(+1\) operator not only creates asymmetry but also enables the exponential shrinking of the numerical structure.

\section{Exclusion of New Cycles}
A new cycle \( C = \{n_1, n_2, \dots, n_k\} \) would need to:
\begin{itemize}
    \item Return to its starting point after \( k \) steps: \( T^k(n) = n \),
    \item Exclude the known cycle \( \{4, 2, 1\} \).
\end{itemize}

However, congruence analysis shows that:
\[
3n + 1 \equiv n \pmod{2^m}
\]
is only possible for \( n \in \{4, 2, 1\} \). For all other numbers, the transformation leads to logarithmic reduction.

The shift introduced by the \(+1\) operator alters the residue class structure modulo \( 2^k \) and prevents a hypothetical transformation \( T^x(n) = n \) from returning exactly to the original state. This distortion makes it mathematically impossible for new stable cycles to emerge.

\subsection{Formal Analysis of Residue Classes mod \( 2^k \)}
To show that no new cycles can arise, we analyze the structure of the transformation \( T(n) = 3n + 1 \) modulo \( 2^k \). For a number \( n \), we define its residue class modulo \( 2^k \) as:
\[
n \equiv r \pmod{2^k}, \quad \text{where } r \in \{0, 1, 2, \dots, 2^k - 1\}.
\]

\subsubsection{Transformation of Odd Numbers:}
   For \( n \) odd, we have:
   \[
   T(n) = 3n + 1.
   \]
   Since \( n \) is odd, it follows that \( 3n \equiv 3 \pmod{2} \). Thus:
   \[
   T(n) \equiv 3n + 1 \pmod{2^k}.
   \]

\subsubsection{Calculation of Residue Classes:}
   The residue class \( r \) of the transformation \( T(n) \) modulo \( 2^k \) is given by:
   \[
   r_{T(n)} \equiv 3r + 1 \pmod{2^k}.
   \]
   This equation shows that each residue class \( r \) is shifted by the operator \( 3r + 1 \).

\subsubsection{Periodicity of Residue Classes:}
   The residue classes \( r \) modulo \( 2^k \) form a cyclic group of order \( 2^k \). The operator \( 3r + 1 \) induces a permutation of the residue classes. To form a cycle, the following must hold:
   \[
   3r + 1 \equiv r \pmod{2^k}.
   \]
   This simplifies to:
   \[
   2r \equiv -1 \pmod{2^k}.
   \]
   For \( r \in \mathbb{N} \) and \( k \geq 2 \), this equation is not solvable because the left-hand side is divisible by \( 2 \), while the right-hand side is odd.

\subsubsection{Exclusion of New Cycles:}
   Since no residue classes \( r \) modulo \( 2^k \) satisfy \( 3r + 1 \equiv r \pmod{2^k} \), no new cycle can arise. Every number is either exponentially reduced or eventually leads to the known cycle \( \{4, 2, 1\} \).

\subsection{Prevention of Hypothetical Symmetry}
Even if a hypothetical cycle were to arise through repeated growth, the transformation remains stable due to asymmetry and logarithmic reduction. The shift caused by the \(+1\) operator prevents a return to symmetry. Mathematically:
\[
\sum_{i=1}^x \frac{+1}{3n_i} \neq 0, \quad \text{for any hypothetical iteration } x.
\]

\section{Systematic Coverage of All Numbers}
The proof encompasses all possible cases:
\begin{itemize}
    \item Small numbers: These can be investigated through direct simulation.
    \item Large numbers: The logarithmic bound guarantees that reduction is unavoidable, even for very large \( n \).
\end{itemize}

\subsection{Long-Term Reduction}
The long-term dynamics of the transformation can be summarized as:
\[
\lim_{k \to \infty} T_k(n) = 0, \quad \text{for all } n \in \mathbb{N}.
\]
This proves that, regardless of the initial number \( n \), there is no possibility of the transformation generating long-term growth or forming new cycles.

\section{Concluding Summary}
The Collatz Conjecture, one of the most fascinating unsolved problems in mathematics, has been examined in this paper through a systematic analysis of the transformation \( T(n) \). The key results and insights can be summarized as follows:

\begin{itemize}
    \item \textbf{Logarithmic Bound:} For every number \( n \in \mathbb{N} \), there exists a \( k \in \mathbb{N} \) such that \( T_k(n) < n \). This bound ensures that every positive integer is reduced after a finite number of steps. It demonstrates how the growth of the \( 3n+1 \) operator is controlled and overcompensated by logarithmic reduction.
    \item \textbf{Exclusion of New Cycles:} A precise congruence analysis has shown that no stable cycles can exist beyond the known cycle \( \{4, 2, 1\} \). This demonstrates that the dynamics of the transformation are fundamentally designed to reduce any growth to a universal cycle.
    \item \textbf{The Role of the \(+1\) Operator:} The \(+1\) operator is central to the structure and dynamics of the transformation. It ensures the asymmetry necessary to transform every odd number into an even number, enabling logarithmic reductions. Without the \(+1\) operator, the transformation would lead to uncontrolled growth.
    \item \textbf{Predictable Dynamics:} Despite the initially chaotic appearance, the Collatz transformation follows clearly defined mathematical rules. The cyclic congruences and bounds allow the progression of every number \( n \in \mathbb{N} \) to be analyzed and predicted until it inevitably enters the cycle \( \{4, 2, 1\} \).
\end{itemize}

The results of this paper provide compelling evidence that the Collatz transformation universally converges to the known cycle \( \{4, 2, 1\} \). The seemingly chaotic dynamics are ordered by deeper mathematical structures, with the role of the \(+1\) operator and the logarithmic bound identified as keys to understanding the conjecture.

\subsection*{Outlook}
This paper not only sheds light on the central dynamics of the Collatz Conjecture but also develops approaches that could be applied to related mathematical problems. Nevertheless, it is essential to view the presented results as a contribution to the mathematical discussion, requiring independent verification and validation by the scientific community.

The methods presented could serve as a foundation for studying similar mathematical transformations characterized by seemingly chaotic behavior. Furthermore, it remains an exciting challenge to conduct numerical simulations over even larger ranges of numbers and to further confirm the universal applicability of the results.

The connection between simple arithmetic and complex mathematical order shows that even seemingly trivial problems can provide deeper insights into the structure of mathematics. The Collatz Conjecture continues to be an inspiring invitation for further research.

\section*{Correction Note}

After further numerical analysis, it has become clear that the initial claim of an exponential reduction requires refinement. The number of reduction steps grows on average with \( O(\log_2 n) \), which describes a logarithmic decrease instead of an exponential one. 

This correction represents a clarification of the original formulation and does not affect the mathematical validity of the work.


\vspace{1cm}
\begin{flushright}
\textit{Stevan Menicanin}
\end{flushright}

\begin{table}[h!]
\centering
\begin{tabular}{|c|p{7cm}|p{5cm}|}
\hline
\textbf{Symbol} & \textbf{Description} & \textbf{Example from the Paper} \\ \hline
\( n \) & Natural number used as input for the Collatz transformation & “Start with any positive integer \( n \).” \\ \hline
\( T(n) \) & The Collatz transformation applied to \( n \) & \( T(n) = \frac{n}{2} \) (for even \( n \)) \\ \hline
\( 3n + 1 \) & Transformation for odd numbers \( n \) & “For odd numbers \( n \), \( T(n) = 3n + 1 \).” \\ \hline
\( \mod \) & Modulo operator, gives the remainder of a division & \( 3n \mod 2 = 1 \), since \( 3n \) for odd \( n \) always leaves a remainder of 1 \\ \hline
\( \pmod{2^k} \) & Residue classes modulo \( 2^k \), used in congruence analysis & “The shift caused by the \(+1\) operator alters the residue class space \( \pmod{2^k} \).” \\ \hline
\( \frac{n}{2} \) & Division by 2, applied to even numbers in the Collatz transformation & \( T(n) = \frac{n}{2} \) \\ \hline
\( T_k(n) \) & Transformation after \( k \) steps & \( T_k(n) = \frac{3n + 1}{2^k} \) \\ \hline
\( P_k \) & Probability that a number is reduced after \( k \) steps & \( P_k = \frac{1}{2^k} \) \\ \hline
\( P_{\text{cumulative}} \) & Cumulative probability that a number is reduced after \( k \) steps & \( P_{\text{cumulative}} = \sum_{i=1}^k \frac{1}{2^i} \) \\ \hline
\( \sum \) & Summation symbol, used for cumulative probability & \( \sum_{i=1}^k \frac{1}{2^i} \) \\ \hline
\( \lim \) & Limit, describes the boundary value of a function & \( \lim_{k \to \infty} P_{\text{cumulative}} = 1 \) \\ \hline
\( \to \) & Indicates convergence or transition to a value & \( \lim_{n \to \infty} \frac{+1}{3n} = 0 \) \\ \hline
\( \neq \) & Not equal, indicates two expressions are not equal & \( \sum_{i=1}^x \frac{+1}{3n_i} \neq 0 \) \\ \hline
\( \log_2 \) & Logarithm to the base 2 & \( k > \log_2(3 + \frac{1}{n}) \) \\ \hline
\( \exists \) & Existential quantifier, indicates the existence of an element & \( \exists k \in \mathbb{N}, \text{ such that } T_k(n) < n \) \\ \hline
\( \in \) & Symbolizes “element of” & \( n \in \mathbb{N} \), \( n \) belongs to the set of natural numbers \\ \hline
\( \mathbb{N} \) & Set of natural numbers & “Every number \( n \in \mathbb{N} \) eventually reaches the cycle \( \{4, 2, 1\}. \)” \\ \hline
\( k \) & Number of transformation steps & “For \( n \to \infty \), the right-hand expression converges to \( \log_2(3) \approx 1.585 \).” \\ \hline
\( \approx \) & Approximately equal & \( \log_2(3) \approx 1.585 \) \\ \hline
\( \{4, 2, 1\} \) & The known cycle of the Collatz transformation & “Every number ends in the cycle \( \{4, 2, 1\}. \)” \\ \hline
\end{tabular}
\caption{Glossary of Mathematical Symbols}
\label{tab:glossary}
\end{table}

\end{document}
