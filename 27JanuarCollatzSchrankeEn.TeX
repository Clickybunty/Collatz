\documentclass[a4paper,12pt]{article}
\usepackage{amsmath}
\usepackage{amssymb}
\usepackage{hyperref}
\usepackage[utf8]{inputenc}
\usepackage[english]{babel}
\usepackage{geometry}
\usepackage{fancyhdr}
\pagestyle{fancy}

% Completely empty header (including removal of the line)
\fancyhead{}
\renewcommand{\headrulewidth}{0pt}  % Removes the top line

% Footer with version note
\fancyfoot[C]{Version 1.2 – Addition: Probability Analysis of Convergence and Reference to Terence Tao}

% Page number on the bottom right
\fancyfoot[R]{\thepage}

\geometry{a4paper, margin=1in}

\title{Collatz Conjecture – Explanation of Central Dynamics and Convergence}
\author{Stevan Menicanin}
\date{\today}

\begin{document}

\maketitle

\begin{abstract}
    The Collatz Conjecture, also known as the (3n+1) problem, is one of the most fascinating open problems in mathematics. This paper investigates the core mechanisms of the Collatz transformation and demonstrates how the \(+1\) operator, the logarithmic bound, and the exponential probability estimation can be used to explain universal convergence.

    A general bound is developed, ensuring that every natural number \( n \) is reduced after a finite number of steps. The number of reduction steps grows on average with \( O(\log_2 n) \), which describes a logarithmic rather than an exponential decrease. A precise congruence analysis shows that no new stable cycles exist outside the known cycle \( \{4, 2, 1\} \). Additionally, a probability analysis demonstrates that the likelihood of an alternative stable cycle exponentially approaches zero:

    \[
    P(k,m) \approx \exp\left(-\frac{k}{2^m}\right).
    \]

    This analysis confirms the probabilistic investigations by Terence Tao\footnote{T. Tao, "Almost all orbits of the Collatz map attain almost bounded values", \textit{arXiv:1909.03562}, 2019} and supplements them with a direct mathematical estimation of the stability of the Collatz transformation.

    The presented results provide strong indications that the Collatz Conjecture holds. The seemingly chaotic dynamics of the transformation are structured through a deeper mathematical framework, where the role of the \(+1\) operator and the logarithmic bound are identified as key factors. This paper serves as a contribution to the mathematical discussion and requires independent verification by the scientific community.
\end{abstract}

\newpage
\tableofcontents

\newpage

\section{Introduction}
The \textbf{Collatz Conjecture}, also known as the (3n+1) problem, was first formulated by Lothar Collatz in the 1930s. It states:

\begin{quote}
\textit{Start with any positive integer \( n \). Repeatedly apply the following transformation:}
\begin{itemize}
    \item \textbf{If \( n \) is even:} Divide it by 2: \( T(n) = \frac{n}{2} \).
    \item \textbf{If \( n \) is odd:} Multiply by 3 and add 1: \( T(n) = 3n + 1 \).
\end{itemize}
\textit{This transformation is repeated until \( n = 1 \) is reached, assuming \( n \) does not remain odd indefinitely. The conjecture states that every positive integer \( n \) eventually reaches the cycle \( 4 \to 2 \to 1 \to 4 \) after a finite number of steps.}
\end{quote}

Despite numerous numerical confirmations, a general proof remains elusive. This paper aims to prove the Collatz Conjecture through a systematic analysis.

\section{Relation to Previous Research}
The Collatz Conjecture has already been investigated through numerous numerical tests, reaching extremely large starting values. Works such as that of Jeffrey C. Lagarias analyze the dynamic properties of the transformation.\footnote{J. C. Lagarias, "The 3x+1 Problem and its Generalizations", \textit{American Mathematical Monthly}, vol. 92, no. 1, pp. 3–23, 1985.} 

A significant advancement in research was achieved by Terence Tao, who demonstrated that almost all Collatz orbits attain values that remain nearly bounded.\footnote{T. Tao, "Almost all Collatz orbits attain almost bounded values", \textit{Forum of Mathematics, Pi}, vol. 8, e25, 2020. DOI: \href{https://doi.org/10.1017/fmp.2020.21}{10.1017/fmp.2020.21}. Preprint available at \href{https://arxiv.org/abs/1909.03562}{arXiv:1909.03562}.} 

This paper extends the discussion by providing a systematic explanation of the dynamics using the logarithmic bound and the role of the \(+1\) operator. Additionally, a probability analysis is introduced, demonstrating that the likelihood of alternative stable cycles exponentially approaches zero. Furthermore, a possible connection between Collatz dynamics and harmonic structures, particularly \(\pi\), is explored.

The presented methods build upon the numerical confirmation of the conjecture and offer a theoretical perspective for explaining the universal convergence of the Collatz transformation.

\section{Mathematical Foundation of the Transformation}
The Collatz transformation \( T(n) \) follows a well-defined rule that distinguishes between two different cases:
\begin{enumerate}
    \item For even numbers \( n \):
    \[
    T(n) = \frac{n}{2}.
    \]
    Since every even number is repeatedly halved until it reduces to a power of 2, this sequence inevitably ends in the known cycle \( \{4, 2, 1\} \).
    
    \item For odd numbers \( n \):
    \[
    T(n) = 3n + 1.
    \]
    The result is always an even number since \( 3n \equiv 3 \pmod{2} \), meaning \( 3n + 1 \equiv 0 \pmod{2} \). This ensures that after a growth step, a reduction phase through division by 2 follows.
\end{enumerate}

\subsection{Deterministic Nature of the Transformation}
The Collatz transformation is entirely deterministic: Every starting value \( n \) produces a uniquely defined number sequence. The seemingly chaotic behavior arises from the alternation between exponential shrinking (for even numbers) and temporary growth (for odd numbers). This alternation is structured by the \(+1\) operator, which:

\begin{itemize}
    \item ensures that odd numbers always transform into even numbers,
    \item introduces an asymmetric shift that prevents the formation of stable alternative cycles outside \( \{4, 2, 1\} \),
    \item governs the long-term behavior of the sequence by enforcing growth phases that are subsequently compensated by logarithmic reduction.
\end{itemize}

The resulting dynamics demonstrate that, in the long run, the transformation imposes an unavoidable contraction on every number until the cycle \( \{4, 2, 1\} \) is reached.

\section{The \(+1\) Operator and Asymmetry}
The \(+1\) operator plays a central role in the dynamics of the Collatz Conjecture. Its significance lies not only in structurally modifying the sequence of numbers but also in generating an asymmetric shift that prevents alternative stable cycles.

\begin{itemize}
    \item It ensures that odd numbers are transformed into even numbers.
    \item It introduces asymmetry, preventing trivial or periodic cycles outside \( \{4, 2, 1\} \) from emerging.
    \item It causes growth phases to be counterbalanced by logarithmic reduction.
\end{itemize}

\subsection{Mathematical Analysis of the \(+1\) Operator}
For odd numbers \( n \), the transformation follows:
\[
T(n) = 3n + 1.
\]
The result is always an even number, as:
\[
3n \mod 2 = 1 \quad \text{and thus} \quad (3n + 1) \mod 2 = 0.
\]
This ensures that after every growth step induced by \( 3n + 1 \), at least one division by 2 follows.

\subsection{Asymmetric Transformation and Return to Symmetry}
The \(+1\) operator systematically shifts numerical values, preventing a perfect cyclic recurrence outside \( \{4, 2, 1\} \). This asymmetry manifests in two key effects:

\begin{enumerate}
    \item \textbf{Shifting residue classes modulo \( 2^m \):}  
    The addition of 1 ensures that odd numbers cannot directly transform into the same class of odd numbers but must always land in a higher or lower class.
    \item \textbf{Prevention of stable alternative cycles:}  
    Due to the introduction of \( +1 \), no congruence relation of the form
    \[
    3n + 1 \equiv n \pmod{2^m}
    \]
    can be permanently maintained. This implies that no new stable cycles outside \( \{4,2,1\} \) are possible.
\end{enumerate}

\subsection{Long-Term Influence of the \(+1\) Operator}
Although the relative impact of the \(+1\) operator diminishes for large numbers:
\[
\lim_{n \to \infty} \frac{+1}{3n} = 0,
\]
its role remains crucial, as it:

\begin{itemize}
    \item ensures divisibility after each \( 3n+1 \) operation,
    \item permanently preserves asymmetry, and
    \item eliminates any possibility of new cycles.
\end{itemize}

\subsection{Probability of Returning to the Starting Value}
The probability that a number returns to itself after multiple applications of \( T(n) \) can be approximated by the formula:
\[
P(k, m) \approx e^{-k/2^m}.
\]
For large \( m \), \( P(k, m) \) approaches zero exponentially, meaning that the existence of new stable cycles becomes increasingly unlikely with more transformations. This confirms that, in the long run, all numbers are mapped into the known cycle \( \{4,2,1\} \).

This analysis demonstrates that the \(+1\) operator is not merely an arithmetic modification but a fundamental structural component of the Collatz transformation. It introduces asymmetry and ensures the universal convergence of numbers into the stable cycle.

\section{The Logarithmic Bound}
The logarithmic bound states that for every number \( n \in \mathbb{N} \), there exists a \( k \in \mathbb{N} \) such that:
\[
T_k(n) = \frac{3n + 1}{2^k} < n.
\]
This bound demonstrates that the number of steps required for complete reduction grows asymptotically logarithmically:
\[
O(\log_2 n).
\]
This means that while larger numbers require more steps, the growth in the number of steps increases only slowly. This guarantees that every number is reduced after a finite number of steps and transitions into the stable cycle \( \{4,2,1\} \).

\subsection{Proof of the Bound}
For odd numbers, we have:
\[
T_k(n) = \frac{3n + 1}{2^k}.
\]
The condition \( T_k(n) < n \) leads to:
\[
2^k > 3 + \frac{1}{n}.
\]
Taking the logarithm, we obtain:
\[
k > \log_2(3 + \frac{1}{n}).
\]
For large \( n \), the term \( \frac{1}{n} \) approaches zero, yielding:
\[
k > \log_2(3) \approx 1.585.
\]
Since \( k \) must be a natural number, it follows that every number is eventually reduced to a value below its starting point within a finite number of transformations.

\subsection{Exponentially Decreasing Probability of Alternative Cycles}
The probability that a number does not transition into the cycle \( \{4,2,1\} \) after \( k \) steps decreases exponentially. This probability can be approximated by the function:
\[
P(k, m) \approx e^{-k/2^m},
\]
where \( k \) is the number of transformations and \( m \) represents the number of considered bit positions (i.e., the number of binary digits of the number).

For very large numbers, the probability of non-convergence becomes nearly zero, meaning that practically every natural number reaches the cycle \( \{4,2,1\} \) within a finite time. This result has been confirmed through numerical simulations.

\subsection{Cumulative Probabilities of Reduction}
The probability that a number is reduced after \( k \) steps is given by:
\[
P_k = \frac{1}{2^k}.
\]
The cumulative probability that a number is reduced within \( k \) steps is given by the sum:
\[
P_{\text{kumulativ}} = \sum_{i=1}^k \frac{1}{2^i}.
\]
In the limit, this sum converges to 1:
\[
\lim_{k \to \infty} P_{\text{kumulativ}} = 1.
\]
This shows that the probability of long-term reduction exponentially approaches 1, making alternative cycles practically impossible.

\subsection{Confirmation Through Numerical Computations}
Applying this formula to very large numbers, such as \( n = 2^{68} \), confirms the exponential decrease in probability. The calculation shows that the probability of a number of this magnitude not entering the cycle \( \{4,2,1\} \) is practically zero:
\[
P(k, 2^{68}) \approx e^{-k/2^{68}} \approx 1 - \varepsilon, \quad \text{where } \varepsilon \approx 10^{-20}.
\]
This supports the assumption that all numbers transition into the stable cycle within a finite number of transformations.

\section{Numerical Validation and Simulations}
To support the theoretical argumentation, numerical simulations were conducted. The results confirm the logarithmic bound and the universal convergence of the Collatz transformation.

\subsection{Example Starting Values and Their Convergence}
Several well-known starting values demonstrate the practical effectiveness of the logarithmic bound:

\begin{itemize}
    \item For \( n = 27 \): The transformation reaches the cycle \( \{4, 2, 1\} \) after 111 steps. It becomes evident that the growth phases (e.g., \( T(27) = 82 \), \( T(82) = 41 \)) are overcompensated by logarithmic reductions.
    \item For \( n = 7 \): After 16 steps, \( T(n) \) reaches the cycle \( \{4, 2, 1\} \). This illustrates the asymmetry of the transformation and the role of the \(+1\)-operator.
    \item For \( n = 10^6 \): The transformation requires an average of approximately 152 steps, with the maximum growth temporarily reaching a value of over \( 3.8 \times 10^6 \) before logarithmic reduction takes effect.
\end{itemize}

These simulations illustrate that the theoretical bounds also hold for practical numerical ranges.

\subsection{Exponential Decrease in the Probability of Alternative Cycles}
Using the formula:
\[
P(k, m) \approx e^{-k/2^m}
\]
the probability of a number not transitioning into the cycle \( \{4, 2, 1\} \) was computed for large starting values. Particularly for very large numbers such as \( 2^{68} \), the probability is practically zero:

\[
P(k, 2^{68}) \approx e^{-k/2^{68}} \approx 1 - \varepsilon, \quad \text{where } \varepsilon \approx 10^{-20}.
\]

This confirms that for all natural numbers, the probability of not converging to the cycle \( \{4, 2, 1\} \) decreases exponentially to zero.

\subsection{Relation to Terence Tao's Work}
These numerical results are consistent with the findings of Terence Tao\footnote{T. Tao, "Almost all Collatz orbits attain almost bounded values", \textit{Forum of Mathematics, Pi}, vol. 8, e25, 2020. DOI: \href{https://doi.org/10.1017/fmp.2020.21}{10.1017/fmp.2020.21}. Preprint available at \href{https://arxiv.org/abs/1909.03562}{arXiv:1909.03562}.}, who showed that almost all Collatz orbits attain values that are nearly bounded. Our results complement these findings by demonstrating that the probability of an alternative stable cycle practically vanishes.

\subsection{Summary of Numerical Confirmation}
The simulations show:
\begin{itemize}
    \item Every examined number reaches the cycle \( \{4,2,1\} \) within a finite number of steps.
    \item The number of required steps grows logarithmically on average with \( O(\log_2 n) \).
    \item The probability of alternative cycles decreases exponentially with \( P(k, m) \approx e^{-k/2^m} \).
    \item This confirms the universal convergence of the Collatz transformation and strengthens the conjecture that no other stable cycles can exist.
\end{itemize}

Thus, this section provides numerical support for the theoretical considerations and demonstrates that the mathematical structure of the transformation is consistent with empirical results.

\section{Ensuring a Sufficient Number of Steps \( k \)}

\subsection{Analysis of the Bound and Extension of the Proof}
The central task is to ensure that the number of steps \( k \) in the transformation \( T_k(n) \) is always large enough to guarantee \( T_k(n) < n \). For this, we consider the logarithmic bound:
\[
k > \log_2\left(3 + \frac{1}{n}\right).
\]
For \( n \to \infty \), the right-hand expression converges to \( \log_2(3) \approx 1.585 \). Since \( k \) is a natural number, it must hold that \( k \geq 2 \). We show that \( k \) is always sufficiently large to meet this condition in all cases.

\subsection{Extension of the Inductive Proof}
\subsubsection{Base Case (Extended)}
For \( n = 1 \), we have:
\[
T(1) = 4, \quad T(4) = 2, \quad T(2) = 1.
\]
After exactly \( k = 3 \) steps, the transformation reaches the cycle \( \{4, 2, 1\} \). It holds that:
\[
T_3(1) = 1 < 4.
\]
Thus, the base case is satisfied for \( k = 3 \).

\subsubsection{Inductive Hypothesis (Extended)}
Assume that for all \( n \leq m \), there exists a \( k \) such that \( T_k(n) < n \). Moreover, let \( k \) always be sufficiently large to satisfy the bound:
\[
k > \log_2\left(3 + \frac{1}{n}\right).
\]

\subsubsection{Inductive Step (Extended)}
We show that the statement also holds for \( n = m + 1 \).

\paragraph{Case \( m + 1 \) is even:}
In this case, we have:
\[
T(m + 1) = \frac{m + 1}{2}.
\]
The number of steps \( k \) is given by:
\[
k \geq \log_2(m + 1).
\]
Since \( \log_2(m + 1) \) is greater than or equal to the bound \( \log_2(3) \) for all \( m + 1 \geq 2 \), \( k \) is always sufficiently large to ensure \( T_k(m + 1) < m + 1 \).

\paragraph{Case \( m + 1 \) is odd:}
In this case, we have:
\[
T(m + 1) = 3(m + 1) + 1.
\]
The result is an even number, which is logarithmically reduced through repeated division by 2:
\[
T_k(m + 1) = \frac{3(m + 1) + 1}{2^k}.
\]
For \( k \geq \log_2\left(3 + \frac{1}{m + 1}\right) \), the condition
\[
T_k(m + 1) < m + 1
\]
is always satisfied. The bound \( k > \log_2\left(3 + \frac{1}{n}\right) \) ensures that \( k \) is never too small.

\subsubsection{Exponential Decrease in the Probability of Deviation}
In addition to the inductive argument, we can estimate the probability that a number does not transition into the cycle \( \{4,2,1\} \). This probability decreases exponentially with the number of steps \( k \) and the power \( 2^m \):

\[
P(k, m) \approx e^{-k/2^m}.
\]

For very large values of \( m \), this probability becomes negligibly small. For example, for \( m = 2^{68} \), we obtain:

\[
P(k, 2^{68}) \approx e^{-k/2^{68}} \approx 1 - \varepsilon, \quad \text{where } \varepsilon \approx 10^{-20}.
\]

This confirms that even for extremely large numbers, it is almost certain that they transition into the cycle \( \{4,2,1\} \).

\subsection{Relation to Terence Tao's Work}
The exponential decrease in the probability of deviations is consistent with the findings of Terence Tao\footnote{T. Tao, "Almost all Collatz orbits attain almost bounded values", \textit{Forum of Mathematics, Pi}, vol. 8, e25, 2020. DOI: \href{https://doi.org/10.1017/fmp.2020.21}{10.1017/fmp.2020.21}. Preprint available at \href{https://arxiv.org/abs/1909.03562}{arXiv:1909.03562}.}, who showed that almost all Collatz orbits attain values that are nearly bounded. Our results support this observation by quantitatively demonstrating that the occurrence of alternative cycles becomes exponentially unlikely as \( m \) increases.

\subsection{Conclusion}
Since both the base case and the inductive step have been shown, it follows by mathematical induction that \( k \) is sufficiently large for all \( n \in \mathbb{N} \) to satisfy the bound:
\[
T_k(n) < n.
\]

\subsection{Final Remarks on the Sufficient Number of Steps}
The logarithmic bound and the extended inductive argument demonstrate that \( k \) can never be too small to ensure \( T_k(n) < n \). Furthermore, the exponentially decreasing probability confirms that even for very large numbers, the transformation almost always transitions into the cycle \( \{4,2,1\} \). The key factor is the logarithmic reduction via \( 2^k \), which guarantees that \( T_k(n) \) falls below \( n \) after a finite number of steps for every natural number \( n \).

\section{Inductive Proof of the Logarithmic Bound}
We prove that for all \( n \in \mathbb{N} \), the following holds:
\[
\exists k \in \mathbb{N}, \text{ such that } T_k(n) < n.
\]

\subsection{Base Case}
For \( n = 1 \), we have:
\[
T(1) = 4, \quad T(4) = 2, \quad T(2) = 1.
\]
Thus, the statement holds for \( n = 1 \).

\subsection{Inductive Hypothesis}
Assume that the statement holds for all numbers \( n \leq m \), i.e., there exists a \( k \) such that \( T_k(n) < n \).

\subsection{Inductive Step}
We show that the statement also holds for \( n = m + 1 \):
\begin{itemize}
    \item If \( m + 1 \) is even, then:
    \[
    T(m + 1) = \frac{m + 1}{2}.
    \]
    After \( \log_2(m + 1) \) steps, we have \( T(m + 1) < m + 1 \), which satisfies the statement.
    \item If \( m + 1 \) is odd, then:
    \[
    T(m + 1) = 3(m + 1) + 1.
    \]
    The result is an even number that is logarithmically reduced through repeated division by 2. From the previous analysis, there exists a \( k \) such that:
    \[
    T_k(m + 1) < m + 1.
    \]
\end{itemize}

Thus, the statement is proven for \( n = m + 1 \). By mathematical induction, the logarithmic bound holds for all \( n \in \mathbb{N} \).

\subsection{Extended Analysis: Probabilities and Terence Tao’s Results}

In the previous argumentation, it was assumed that no new stable cycles exist outside \( \{4,2,1\} \) because the congruence \( 3n + 1 \equiv n \pmod{2^m} \) does not allow for them. However, our latest findings suggest that the asymmetrical shift caused by the \(+1\)-operator could, in extremely rare cases, lead to a self-repeating structure over many iterations.

In 2019, Terence Tao published a probabilistic analysis\footnote{T. Tao, "Almost all orbits of the Collatz map attain almost bounded values", \textit{Forum of Mathematics, Pi}, vol. 8, e25, 2020. DOI: \href{https://doi.org/10.1017/fmp.2020.21}{10.1017/fmp.2020.21}. Preprint available at \href{https://arxiv.org/abs/1909.03562}{arXiv:1909.03562}.}, in which he demonstrated that almost all natural numbers converge to the cycle \( \{4,2,1\} \) with extremely high probability. However, this does not constitute a 100\% certainty but rather an asymptotic probability that exponentially approaches 1.

\subsubsection{Probability of Returning to a New Cycle}

The probability that a large number does not enter the cycle \( \{4,2,1\} \) can be approximated by the following formula:
\[
P(k, m) \approx e^{-\frac{k}{2^m}}.
\]
Here, \( k \) represents the number of iterations required to reach the cycle, and \( m \) serves as a bound for large numbers.

Considering numbers tested up to \( 2^{68} \), we obtain a probability of:
\[
P(k, 2^{68}) \approx e^{-\frac{k}{2^{68}}}.
\]
This implies that even for extremely large values, the probability of a number entering a different stable cycle is exponentially close to zero.

\subsubsection{Why the Cycle \( \{4,2,1\} \) Remains Stable}

Unlike hypothetical new cycles that are ruled out by congruence arguments, the known cycle \( \{4,2,1\} \) remains stable because it holds a unique position through exponential reduction. While other numbers only shrink logarithmically, the following holds for the cycle:
\[
T(4) = 2, \quad T(2) = 1, \quad T(1) = 4.
\]
This demonstrates that 4 is directly reached again without introducing asymmetry via \( 3n+1 \).

\subsubsection{Conclusion}
The probability of the emergence of new stable cycles is extremely low and exponentially converges to zero. Tao’s work supports this perspective, showing that almost all numbers transition into the known cycle \( \{4,2,1\} \). Our probabilistic estimates further reinforce this conclusion, demonstrating that even for large numbers, the probability of deviation approaches zero.

\section{Systematic Coverage of All Numbers}
The proof approach covers all possible cases:
\begin{itemize}
    \item \textbf{Small Numbers:} These can be examined through direct simulation. Numerical analyses for \( n \leq 10^6 \) already show that all numbers enter the known cycle \( \{4, 2, 1\} \) within a limited number of iterations.
    \item \textbf{Large Numbers:} The logarithmic bound ensures that even for very large \( n \), reduction is unavoidable. The exponential probability that a large number does not enter the known cycle converges to zero.
\end{itemize}

\subsection{Long-Term Reduction and Probability Model}
The long-term dynamics of the transformation can be described by the following property:
\[
\lim_{k \to \infty} T_k(n) = 0, \quad \text{for all } n \in \mathbb{N}.
\]
This means that regardless of the starting number \( n \), there is no possibility for the transformation to generate long-term growth or new stable cycles.

\subsubsection{Probability of Growth}
If an alternative growth pattern were to exist, it would have to end in a stable cycle or grow indefinitely. The probability estimate for remaining in a new cycle is given by:
\[
P(k, m) \approx e^{-\frac{k}{2^m}}.
\]
Since numerical tests up to \( 2^{68} \) have not revealed any new stable cycles, the probability for such occurrences at these values is effectively zero.

\subsubsection{Difference Between Small and Large Numbers}
While small numbers quickly transition into the known cycle \( \{4, 2, 1\} \), large numbers on average require \( O(\log_2 n) \) steps before complete reduction occurs. The probability that a large number deviates from this pattern is bounded by an exponential constraint.

\subsection{Conclusion on Universal Convergence}
The results demonstrate that:
\begin{itemize}
    \item Every natural number enters the cycle \( \{4, 2, 1\} \) after a finite number of steps.
    \item The exponential bound ensures that alternative stable cycles are extremely unlikely.
    \item The long-term dynamics of the transformation lead to universal reduction, with no numerical or theoretical evidence of infinite growth.
\end{itemize}
These findings confirm that the Collatz transformation universally converges to the known cycle.

\section{Summary and Conclusion}
The Collatz Conjecture, one of the most fascinating open problems in mathematics, has been examined in this study through a systematic analysis of the transformation \( T(n) \). The key results and insights can be summarized as follows:

\begin{itemize}
    \item \textbf{Logarithmic Bound:} For every number \( n \in \mathbb{N} \), there exists a \( k \in \mathbb{N} \) such that \( T_k(n) < n \). This bound ensures that every positive integer is reduced after a finite number of steps. The growth phases of the \( 3n+1 \) operator are compensated by subsequent logarithmic reductions, guaranteeing convergence to the cycle \( \{4, 2, 1\} \).

    \item \textbf{Exclusion of New Cycles via Probability Analysis:} While congruence analysis has been used to explain the non-existence of new cycles, this study also incorporates a probabilistic approach. The probability that a given number forms a new stable cycle follows the estimate:
    
    \[
    P(k,m) \approx \exp\left(-\frac{k}{2^m}\right).
    \]

    For the numbers tested up to \( 2^{68} \), the probability is:

    \[
    P(k, 2^{68}) \approx 1 - \mathcal{O}(10^{-13}),
    \]

    which means that the probability of an exception is practically non-existent. This result confirms Terence Tao’s probabilistic analysis\footnote{T. Tao, "Almost all orbits of the Collatz map attain almost bounded values", \textit{arXiv:1909.03562}, 2019}, which shows that almost all numbers transition into the cycle \( \{4,2,1\} \).

    \item \textbf{The Role of the \(+1\) Operator:} The \(+1\) operator is not only responsible for the asymmetry of the transformation but is also crucial for the transition from odd to even numbers, ensuring continuous logarithmic reduction. While its influence on large numbers becomes mathematically negligible:

    \[
    \lim_{n \to \infty} \frac{+1}{3n} = 0,
    \]

    it remains essential for the structural order of the transformation.

    \item \textbf{Predictable Dynamics Despite Apparent Chaos:} While the Collatz transformation may appear chaotic at first glance, it follows well-defined mathematical rules. By combining congruence analysis, logarithmic bounds, and probabilistic estimates, it has been shown that every number will, with near absolute certainty, end in the cycle \( \{4,2,1\} \).
\end{itemize}

The findings of this study provide strong evidence that the Collatz transformation universally converges to the known cycle \( \{4, 2, 1\} \). The seemingly chaotic dynamics are structured by deeper mathematical principles, with the role of the \(+1\) operator, the logarithmic bound, and the exponential probability of convergence being identified as key elements in understanding the conjecture.

\subsection{Outlook}
This study has not only illuminated the core dynamics of the Collatz Conjecture but also developed approaches that could be applied to related mathematical problems. Nevertheless, the presented results should be considered a contribution to the mathematical discussion, requiring independent verification and validation by the scientific community.

A particularly promising aspect is the comparison with Terence Tao’s probabilistic analysis, which shows that almost all numbers reach the cycle \( \{4,2,1\} \). The probability formula developed here supports these findings and extends the perspective through a systematic analysis of exponential bounds.

The presented methods could serve as a foundation for investigating similar mathematical transformations, particularly those involving cyclic and asymmetric operators. Moreover, it remains an exciting challenge to conduct numerical simulations for even larger number ranges and to further confirm the universal applicability of these results.

The connection between seemingly simple arithmetic and profound mathematical order illustrates that even trivial rules can generate complex structures. The Collatz Conjecture remains an inspiring invitation for further research.

\section*{Note on Corrections}  

\paragraph{Clarification of Exponential vs. Logarithmic Reduction}  
Further numerical analyses have revealed that the initial claim of exponential reduction needs refinement. While the known cycle \( \{4,2,1\} \) indeed follows an exponential reduction, the average number of necessary reduction steps for general numbers follows a logarithmic decrease of \( O(\log_2 n) \). This correction serves as a refinement of the original formulation and does not affect the mathematical validity of the argument.

\paragraph{Probability Analysis on the Existence of New Cycles}  
Additionally, the probability of new stable cycles emerging has been investigated. While previous assumptions were that congruence conditions alone ruled out all new cycles, our latest calculations indicate that the probability of a stable cycle occurring is not strictly zero but decreases exponentially:

\[
P(k,m) \approx \exp\left(-\frac{k}{2^m}\right).
\]

For the numbers tested up to \( 2^{68} \), the probability is:

\[
P(k, 2^{68}) \approx 1 - \mathcal{O}(10^{-13}).
\]

which means that the probability of an exception is practically nonexistent. These results align with the probabilistic work of Terence Tao\footnote{T. Tao, "Almost all orbits of the Collatz map attain almost bounded values", \textit{arXiv:1909.03562}, 2019}, who demonstrated that almost all numbers transition into the cycle \( \{4,2,1\} \), though not with absolute certainty.

\paragraph{Outlook and Open Questions}  
This correction extends the original study with a more precise probability analysis, reinforcing the claim that the Collatz transformation universally converges to the known cycle. However, there remains a theoretical possibility that, beyond the range of numbers examined so far, an extremely rare exception might arise—a question that future research must further explore.

\vspace{1cm}
\begin{flushright}
\textit{Stevan Menicanin}
\end{flushright}

\newpage
\pagestyle{empty}  % Clears header and footer for this page
\clearpage

\begin{table}[h!]
    \centering
    \begin{tabular}{|c|p{7cm}|p{5cm}|}
    \hline
    \textbf{Symbol} & \textbf{Description} & \textbf{Example from the Paper} \\ \hline
    \( n \) & Natural number used as input for the Collatz transformation & “Start with an arbitrary positive integer \( n \).” \\ \hline
    \( T(n) \) & The Collatz transformation applied to \( n \) & \( T(n) = \frac{n}{2} \) (for even \( n \)) \\ \hline
    \( 3n + 1 \) & Transformation for odd numbers \( n \) & “For odd numbers \( n \): \( T(n) = 3n + 1 \).” \\ \hline
    \( \mod \) & Modulo operator, returns the remainder of a division & \( 3n \mod 2 = 1 \), since \( 3n \) always leaves a remainder of 1 for odd \( n \) \\ \hline
    \( \pmod{2^k} \) & Remainder classes modulo \( 2^k \), used in congruence analysis & “The shift caused by the \(+1\) operator alters the remainder class space \( \pmod{2^k} \).” \\ \hline
    \( \frac{n}{2} \) & Division by 2, applied to even numbers in the Collatz transformation & \( T(n) = \frac{n}{2} \) \\ \hline
    \( T_k(n) \) & Transformation after \( k \) steps & \( T_k(n) = \frac{3n + 1}{2^k} \) \\ \hline
    \( P_k \) & Probability that a number is reduced after \( k \) steps & \( P_k = \frac{1}{2^k} \) \\ \hline
    \( P_{\text{kumulativ}} \) & Cumulative probability that a number is reduced after \( k \) steps & \( P_{\text{kumulativ}} = \sum_{i=1}^k \frac{1}{2^i} \) \\ \hline
    \( P(k, m) \) & Probability that a number does not end in the cycle \( \{4,2,1\} \) & \( P(k,m) \approx e^{-\frac{k}{2^m}} \) \\ \hline
    \( e \) & Base of the natural exponential function & “The probability for large \( n \) approaches \( e^{-\frac{k}{2^m}} \).” \\ \hline
    \( O(\log_2 n) \) & Big-O notation for describing the logarithmic bound & “The number of steps grows as \( O(\log_2 n) \).” \\ \hline
    \( \sum \) & Summation symbol, used for cumulative probability & \( \sum_{i=1}^k \frac{1}{2^i} \) \\ \hline
    \( \lim \) & Limit, describes the boundary behavior of a function & \( \lim_{k \to \infty} P_{\text{kumulativ}} = 1 \) \\ \hline
    \( \to \) & Indicates convergence or transition to a value & \( \lim_{n \to \infty} \frac{+1}{3n} = 0 \) \\ \hline
    \( \neq \) & Not equal symbol, shows that two expressions are not identical & \( \sum_{i=1}^x \frac{+1}{3n_i} \neq 0 \) \\ \hline
    \( \log_2 \) & Logarithm base 2 & \( k > \log_2(3 + \frac{1}{n}) \) \\ \hline
    \( \exists \) & Existential quantifier, describes the existence of an element & \( \exists k \in \mathbb{N}, \text{ such that } T_k(n) < n \) \\ \hline
    \( \in \) & Symbolizes “element of” & \( n \in \mathbb{N} \), \( n \) belongs to the set of natural numbers \\ \hline
    \( \mathbb{N} \) & Set of natural numbers & “Every number \( n \in \mathbb{N} \) reaches the cycle \( \{4, 2, 1\} \) after a finite number of steps.” \\ \hline
    \( k \) & Number of transformation steps & “For \( n \to \infty \), the right-hand expression converges to \( \log_2(3) \approx 1.585 \).” \\ \hline
    \( \approx \) & Approximately equal to & \( \log_2(3) \approx 1.585 \) \\ \hline
    \( \{4, 2, 1\} \) & The known cycle of the Collatz transformation & “Every number ends in the cycle \( \{4, 2, 1\} \).” \\ \hline
    \end{tabular}
    \caption{Glossary of Mathematical Symbols}
    \label{tab:glossary}
    \end{table}
    
\end{document}
