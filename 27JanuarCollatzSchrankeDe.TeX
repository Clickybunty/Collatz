\documentclass[a4paper,12pt]{article}
\usepackage{amsmath}
\usepackage{amssymb}
\usepackage{hyperref}
\usepackage[utf8]{inputenc}
\usepackage[ngerman]{babel}
\usepackage{geometry}
\usepackage{fancyhdr}
\usepackage{graphicx}
\usepackage{pgfplots}
\usepackage{tikz}

\geometry{a4paper, margin=1in}
\pagestyle{fancy}

\fancyhead{}
\renewcommand{\headrulewidth}{0pt}  
\fancyfoot[C]{Version 3.1 – Präzisierte Asymmetrieanalyse, strukturelle Schranke von $d(n)$ und Multiplikations-Divisions-Verhältnis}

\fancyfoot[R]{\raisebox{-5mm}{\thepage}}

\title{Collatz-Vermutung – Mathematische Struktur und universelle Konvergenz}
\author{Stevan Menicanin}
\date{\today}

\begin{document}

\maketitle

\section*{Anmerkung zur Revision (Version 3.1)}
Diese überarbeitete Version ersetzt die vorherige vollständig. Frühere Versionen verwendeten eine Wahrscheinlichkeitsabschätzung zur Existenz neuer stabiler Zyklen. Neue Erkenntnisse zur Distanzfunktion $d(n)$ zeigen jedoch, dass $2n$ niemals in der Collatz-Folge von $n$ auftritt, wodurch die Wahrscheinlichkeitsannahme obsolet wird und neue Zyklen strukturell ausgeschlossen sind.

Zusätzlich wurde die Analyse des Multiplikations- und Divisionsverhältnisses überarbeitet. Die ursprüngliche theoretische Abschätzung von $m/d \approx 1.261$ hat sich in der Simulation nicht bestätigt. Stattdessen zeigt sich ein systematisches Verhältnis von $m/d \approx 0.639$, das durch die direkte Kompensation von Multiplikationen mit anschließenden Divisionen sowie durch den Einfluss des $+1$-Operators erklärt werden kann. Diese strukturelle Asymmetrie verhindert eine exakte Rückkehr zu $2n$ und bestätigt die empirisch gefundene Schranke von $d(n) \geq 0.00418$.

Die Argumentation wurde daher überarbeitet und basiert nun vollständig auf einer deterministischen Beweisführung unter Verwendung von $d(n)$ und der strukturellen Analyse des Multiplikations- und Divisionsverhältnisses. Diese Version verstärkt den Konvergenzbeweis, indem sie die Notwendigkeit probabilistischer Annahmen eliminiert und die strukturelle Stabilität der Collatz-Transformation formal begründet.

\begin{abstract}
    Die Collatz-Vermutung ist eines der faszinierendsten offenen Probleme der Mathematik. Diese Arbeit präsentiert eine neue Herangehensweise zur Erklärung der universellen Konvergenz durch eine detaillierte Analyse der Transformation $T(n)$. 
    
    Ein zentrales Element ist die Distanzfunktion $d(n)$, die den minimalen Abstand zwischen $2n$ und einer Zahl in der Collatz-Folge von $n$ beschreibt. Es wird gezeigt, dass $d(n)$ streng monoton wächst und niemals Null erreicht, wodurch neue stabile Zyklen ausgeschlossen sind.

    Neben der Distanzfunktion wird das Verhältnis von Multiplikationen und Divisionen in der Collatz-Folge untersucht. Theoretisch wäre ein Verhältnis von $m/d \approx 1.261$ erforderlich, um exakt 200\% des Ausgangswertes zu erreichen. Die Simulation zeigt jedoch eine systematische Abweichung mit $m/d \approx 0.639$, was zu einer strukturellen Schranke führt, die eine Rückkehr zu $2n$ verhindert.

    Im Gegensatz zu probabilistischen Ansätzen basiert dieser Beweis auf einer deterministischen Analyse der mathematischen Struktur. Die logarithmische Schranke für die notwendige Anzahl von Reduktionsschritten wird formal bewiesen, und die strukturelle Asymmetrie zwischen Multiplikation und Division wird als zentrales Element für die Konvergenz identifiziert.

    Diese Arbeit belegt, dass die Collatz-Transformation keine stabilen Zyklen außerhalb von $\{4, 2, 1\}$ zulässt. Damit wird die universelle Konvergenz der Collatz-Folge durch eine strukturelle Argumentation gestützt, die probabilistische Annahmen ersetzt.
\end{abstract}



\newpage
\tableofcontents

\clearpage





\section{Einleitung}
Die \textbf{Collatz-Vermutung}, auch bekannt als das (3n+1)-Problem, wurde in den 1930er Jahren von Lothar Collatz formuliert und zählt zu den faszinierendsten ungelösten Problemen der Mathematik. Die zugrunde liegende Transformation ist einfach definiert:

\begin{quote}
\textit{Beginnend mit einer beliebigen positiven ganzen Zahl \( n \) wird wiederholt folgende Regel angewandt:}
\begin{itemize}
    \item \textbf{Falls \( n \) gerade ist:} \( T(n) = \frac{n}{2} \).
    \item \textbf{Falls \( n \) ungerade ist:} \( T(n) = 3n + 1 \).
\end{itemize}
\textit{Die Vermutung besagt, dass jede positive Zahl \( n \) nach endlich vielen Iterationen in den Zyklus \( \{4, 2, 1\} \) übergeht.}
\end{quote}

Diese Arbeit untersucht die mathematische Struktur der Collatz-Transformation und liefert eine neue Beweisführung, die zeigt, dass alle natürlichen Zahlen in den bekannten Zyklus überführt werden und keine weiteren stabilen Zyklen existieren.



\section{Bezug zur bisherigen Forschung}
Die Collatz-Vermutung wurde durch umfangreiche numerische Tests mit extrem großen Startwerten untersucht. Besonders hervorzuheben sind die Arbeiten von Jeffrey C. Lagarias, der die dynamischen Eigenschaften der Transformation analysierte und wichtige Grundlagen für das Verständnis der strukturellen Muster der Collatz-Folge legte.\footnote{J. C. Lagarias, "The 3x+1 Problem and its Generalizations", \textit{American Mathematical Monthly}, vol. 92, no. 1, pp. 3–23, 1985.} 

Ein weiterer bedeutender Beitrag stammt von Terence Tao, der eine tiefgehende Untersuchung der Collatz-Orbits durchführte. Seine Analyse zeigt, dass fast alle Startwerte eine gewisse Regularität aufweisen, jedoch ohne eine strikte Konvergenzgarantie.\footnote{T. Tao, "Almost all Collatz orbits attain almost bounded values", \textit{Forum of Mathematics, Pi}, vol. 8, e25, 2020. DOI: \href{https://doi.org/10.1017/fmp.2020.21}{10.1017/fmp.2020.21}. Preprint verfügbar unter \href{https://arxiv.org/abs/1909.03562}{arXiv:1909.03562}.} Während frühere Arbeiten häufig probabilistische Argumente nutzten, um die Stabilität der Collatz-Transformation zu beschreiben, verfolgt diese Arbeit einen anderen Ansatz.

Hier wird die probabilistische Betrachtung durch eine deterministische strukturelle Analyse ersetzt. Der zentrale Ansatz basiert auf der Distanzfunktion \( d(n) \), die zeigt, dass \( 2n \) niemals Teil der Collatz-Folge von \( n \) sein kann. Dadurch wird der Ausschluss alternativer stabiler Zyklen nicht nur als unwahrscheinlich, sondern als strukturell unmöglich belegt.

Diese neue Perspektive führt zu einer wesentlichen Verschärfung bisheriger Konvergenzargumente. Da alternative Zyklen durch eine inhärente Eigenschaft der Collatz-Transformation ausgeschlossen sind, entfällt die Notwendigkeit probabilistischer Modelle vollständig.







\section{Asymmetrie des \( +1 \)-Operators und die Schranke von \( d(n) \)}

Die Collatz-Transformation basiert auf zwei grundlegenden Operationen:

\begin{itemize}
    \item Reduktion durch Division bei geraden Zahlen.
    \item Wachstum durch den \( 3n+1 \)-Operator bei ungeraden Zahlen.
\end{itemize}

Während der Einfluss des \( +1 \)-Operators für kleine \( n \) signifikant ist, nimmt er mit steigendem \( n \) stetig ab. Gleichzeitig wächst die Distanzfunktion \( d(n) \), was strukturell ausschließt, dass alternative stabile Zyklen existieren können.

\subsection{Mathematische Analyse des \( +1 \)-Operators}

Der \( +1 \)-Operator erzeugt eine minimale Asymmetrie, indem er ungerade Zahlen zu geraden transformiert. Sein relativer Einfluss auf \( n \) ergibt sich durch den Grenzwert:

\begin{equation} 
    \lim_{n \to \infty} \frac{+1}{3n} = 0.
\end{equation}

Dies zeigt, dass für große \( n \) die additive Modifikation durch \( +1 \) vernachlässigbar wird, insbesondere im Vergleich zur Multiplikation mit 3.

Numerische Analysen zeigen, dass der Einfluss des \( +1 \)-Operators für \( n < 10^4 \) noch messbar ist und einzelne Werte unterhalb der Schranke \( d(n) \geq 0.00418 \) fallen können. Für \( n > 10^4 \) liegt dieser Einfluss unter 0.0033\% und ist vernachlässigbar.

\subsection{Wachstum der Distanzfunktion \( d(n) \) und ihre Schranke}

Die Distanzfunktion \( d(n) \) beschreibt den minimalen Abstand zwischen \( 2n \) und einer Zahl innerhalb der Collatz-Folge von \( n \). Numerische Analysen zeigen, dass \( d(n) \) mit wachsendem \( n \) streng monoton wächst und sich durch die Näherungsformel

\begin{equation} 
    d(n) \geq 0.00418
\end{equation}

beschreiben lässt. Diese Schranke ergibt sich aus der Simulation der Multiplikations- und Divisionsvorgänge in der Collatz-Folge:

\begin{equation}
    3^m \cdot 2^{-d} \approx 2.00418.
\end{equation}

Da der Einfluss des \( +1 \)-Operators für große \( n \) verschwindet, bleibt diese Schranke stabil. Die empirischen Daten bestätigen, dass \( d(n) \) für große \( n \) nicht mehr unterhalb dieser Schranke fällt.

\subsection{Wechselwirkung zwischen \( +1 \)-Operator und Distanzfunktion}

Der Einfluss des \( +1 \)-Operators nimmt für große Zahlen stetig ab, während \( d(n) \) parallel dazu wächst. Dies führt zu folgender Beziehung:

\begin{equation} 
    \lim_{n \to \infty} \frac{+1}{3n} = 0 \quad \text{und} \quad \lim_{n \to \infty} d(n) \geq 0.00418.
\end{equation}

Damit zeigt sich, dass der \( +1 \)-Operator langfristig keinen entscheidenden Einfluss mehr hat, während \( d(n) \) strukturell verhindert, dass \( 2n \) jemals in der Collatz-Folge von \( n \) auftritt. Diese Eigenschaften schließen alternative stabile Zyklen aus.



\section{Simulation und Herleitung des Multiplikations- und Divisionsverhältnisses}  

Ein zentraler Aspekt der Collatz-Transformation ist das Zusammenspiel von Multiplikationen mit 3 und Divisionen durch 2. Die folgende Analyse zeigt, welches Verhältnis dieser beiden Operationen notwendig wäre, um exakt das Doppelte des Ausgangswertes zu erreichen, und warum die empirischen Ergebnisse eine Abweichung davon aufweisen.  

\subsection{Theoretische Herleitung des Verhältnisses}  

Um zu untersuchen, wie sich eine Zahl in der Collatz-Folge verändert, betrachten wir die Anzahl der Multiplikationen mit 3 (\( m \)) und Divisionen mit 2 (\( d \)), die notwendig wären, um exakt 200\% des Ausgangswerts zu erreichen.  

Jeder Multiplikationsschritt \( 3^m \) vergrößert den Wert um einen Faktor \( 3^m \), während jede Division mit 2 (\( 2^{-d} \)) den Wert um einen Faktor \( 2^{-d} \) verkleinert. Damit der Wert genau das Doppelte des Ausgangswertes erreicht, muss gelten:  

\begin{equation}
    3^m \cdot 2^{-d} = 2
\end{equation}

Durch Logarithmierung erhalten wir:

\begin{equation}
    \log(3^m) - \log(2^d) = \log(2)
\end{equation}

Mit der Logarithmeneigenschaft \( \log(a^b) = b \cdot \log(a) \) folgt:

\begin{equation}
    m \cdot \log(3) - d \cdot \log(2) = \log(2)
\end{equation}

Nach Umstellen ergibt sich:

\begin{equation}
    \frac{m}{d} = \frac{\log(2)}{\log(3) - \log(2)}
\end{equation}

Setzen wir die Werte \( \log(2) \approx 0.3010 \) und \( \log(3) \approx 0.4771 \) ein:

\begin{equation}
    \frac{m}{d} = \frac{0.3010}{0.4771 - 0.3010} = \frac{0.3010}{0.1761} \approx 1.261
\end{equation}

Das bedeutet:  
Um exakt 200\% zu erreichen, müsste das Verhältnis von Multiplikationen zu Divisionen \( m/d \approx 1.261 \) betragen.  

\subsection{Abweichung in der Simulation}  

Unsere Simulation zeigt jedoch, dass die tatsächlich gemessene Anzahl von Multiplikationen im Verhältnis zu den Divisionen nicht exakt 1.261 beträgt. Stattdessen ergibt sich:

\begin{equation}
    \frac{m}{d} \approx 0.639
\end{equation}

Dadurch wird nicht exakt \( 2.00000 \) erreicht, sondern:

\begin{equation}
    3^m \cdot 2^{-d} \approx 2.00418
\end{equation}

Diese Abweichung erklärt die zuvor gefundene minimale Distanzfunktion \( d(n) \) und bestätigt, dass eine exakte Rückkehr zu \( 2n \) strukturell ausgeschlossen ist.  

\subsection{Warum das optimale Verhältnis nicht erreicht wird}  

Das theoretische Verhältnis \( m/d \approx 1.261 \) kann in der Realität nicht erreicht werden, da zwei zentrale Mechanismen der Collatz-Folge dies verhindern:

\begin{enumerate}
    \item \textbf{Jede Multiplikation wird sofort durch eine Division kompensiert:}  
    Sobald eine ungerade Zahl \( n \) auftritt, wird sie durch den \( 3n+1 \)-Operator in eine gerade Zahl umgewandelt. Da jede gerade Zahl durch \( 2 \) teilbar ist, folgt direkt eine Division durch 2.  
    \textbf{Ergebnis:} Nach jeder Multiplikation mit 3 folgt mindestens eine Division, wodurch das Verhältnis \( m/d \) sinkt.

    \item \textbf{Der \( +1 \)-Operator verschiebt die Werte zusätzlich:}  
    Der Term \( 3n+1 \) sorgt dafür, dass die Zahlen auf eine neue Restklasse modulo \( 2^k \) verschoben werden, wodurch die Verteilung der Divisionen ungleichmäßig wird. Dadurch werden zusätzliche Divisionen notwendig, die das Verhältnis weiter verringern.
\end{enumerate}

Da sowohl der direkte Ausgleich durch eine Division als auch die strukturelle Verschiebung durch den \( +1 \)-Operator eine Rolle spielen, bleibt das gemessene Verhältnis bei etwa \( m/d \approx 0.639 \) statt der erwarteten \( 1.261 \).  

\subsection{Zusammenhang mit der Schranke}  

Die Theorie sagt, dass ein Verhältnis von \( m/d = 1.261 \) exakt 200\% (\( 2n \)) ergeben würde.  
In der Praxis messen wir jedoch \( m/d \approx 0.639 \), was dazu führt, dass der Wert immer knapp über 2 liegt.  

Der Überschuss von \( 0.00418 \) stellt genau die Schranke dar, die wir in der Wachstumsanalyse von \( d(n) \) gefunden haben. Damit bestätigt sich, dass die Asymmetrie zwischen Multiplikation und Division strukturell verhindert, dass sich ein Wert zurück auf \( 2n \) entwickelt.  

\subsection{Fazit}  

\begin{itemize}
    \item Die theoretische Formel \( 3^m \cdot 2^{-d} = 2 \) beschreibt den idealisierten Fall, in dem genau 200\% erreicht werden.  
    \item Die Simulation zeigt jedoch eine systematische Abweichung, sodass in der Realität \( 3^m \cdot 2^{-d} \approx 2.00418 \) gilt.  
    \item Die minimale Differenz von \( 0.00418 \) ist genau die empirisch bestätigte Schranke, die eine zyklische Rückführung zu \( 2n \) verhindert.  
    \item Das optimale Verhältnis von \( m/d \approx 1.261 \) kann nicht erreicht werden, da jede Multiplikation sofort durch eine Division kompensiert wird und der \( +1 \)-Operator zusätzliche Verschiebungen verursacht.  
    \item Dieses Ungleichgewicht zwischen Multiplikation und Division zeigt, dass die Collatz-Transformation strukturell asymmetrisch ist und keine perfekten Rückkehrpunkte zu \( 2n \) zulässt.  
\end{itemize}




\subsection{Folgerungen für die Collatz-Vermutung}

Da \( d(n) \) für alle untersuchten Zahlen wächst, folgt, dass \( 2n \) niemals Teil der Collatz-Folge von \( n \) sein kann. Daraus ergeben sich die folgenden Schlussfolgerungen:

\begin{itemize}
    \item Der \( +1 \)-Operator verliert für große \( n \) an Bedeutung.
    \item Der Abstand \( d(n) \) wächst so stark, dass \( 2n \) niemals zurückgeführt werden kann.
    \item Kein Wert kann durch eine alternative Iteration auf sich selbst zurückfallen.
\end{itemize}

\textbf{Fazit:} Der abnehmende Einfluss des \( +1 \)-Operators und das zunehmende Wachstum von \( d(n) \) sind zwei miteinander verknüpfte Eigenschaften der Collatz-Transformation. Ihre symmetrische Entwicklung bestätigt, dass alternative stabile Zyklen strukturell ausgeschlossen sind.





\section{Mathematische Grundlage der Transformation}

Die Collatz-Transformation \( T(n) \) folgt einer rekursiven Regel mit zwei Fällen:
\begin{enumerate}
    \item Für gerade Zahlen \( n \) gilt:
    \[
    T(n) = \frac{n}{2}.
    \]
    Da jede gerade Zahl durch wiederholtes Halbieren auf eine Potenz von 2 reduziert wird, endet diese Sequenz zwangsläufig im Zyklus \( \{4, 2, 1\} \).
    
    \item Für ungerade Zahlen \( n \) gilt:
    \[
    T(n) = 3n + 1.
    \]
    Da \( 3n \equiv 3 \pmod{2} \) ist, bleibt \( 3n+1 \) stets gerade, sodass nach einem Wachstumsschritt eine Reduktionsphase durch Division mit 2 folgt. Die ungeraden Werte bestimmen somit die Struktur der Transformation.
\end{enumerate}

\subsection{Deterministische Natur der Transformation}

Die Collatz-Transformation ist vollständig deterministisch: Jeder Startwert \( n \) erzeugt eine eindeutig definierte Zahlenfolge. Ihr scheinbar chaotisches Verhalten resultiert aus dem Wechsel zwischen Wachstum und Reduktion.

Der \( +1 \)-Operator spielt dabei eine zentrale Rolle:
\begin{itemize}
    \item Er stellt sicher, dass ungerade Zahlen in gerade Zahlen überführt werden.
    \item Er erzeugt eine Asymmetrie, die das Wachstum steuert.
    \item In Kombination mit der Division bewirkt er eine langfristige Schrumpfung.
\end{itemize}



\subsection{Erweiterung der Asymmetrieanalyse durch die Distanzfunktion \( d(n) \)}

Neben der kongruenztheoretischen Betrachtung des \(+1\)-Operators ermöglicht die Distanzfunktion \( d(n) \) eine zusätzliche Analyse der strukturellen Asymmetrie der Collatz-Transformation. Sie beschreibt den minimalen Abstand zwischen \( 2n \) und einer Zahl in der Collatz-Folge von \( n \):

\begin{equation}
    d(n) = \min_{x \in \text{Collatz-Folge}(n)} |x - 2n|.
\end{equation}

Unsere Simulationsergebnisse zeigen, dass die Multiplikation mit 3 und die Division durch 2 eine untere Schranke für \( d(n) \) ergeben:

\begin{equation}
    3^m \cdot 2^{-d} \approx 2.00418,
\end{equation}

was bestätigt, dass eine exakte Rückkehr zu \( 2n \) ausgeschlossen ist.

Empirische Untersuchungen von Zahlen bis \( 50.000.000 \) zeigen, dass sich \( d(n) \) mit wachsendem \( n \) durch die Näherungsformel

\begin{equation}
    d(n) \geq 0.00418 \cdot n
\end{equation}

beschreiben lässt. Diese Funktion bleibt für alle untersuchten Werte stabil und bestätigt, dass sich \( d(n) \) mit steigendem \( n \) nicht unter diese Schranke reduziert.



\subsection{Interpretation der Distanzfunktion \( d(n) \)}

Die Analyse von \( d(n) \) zeigt zwei zentrale Muster:

\begin{enumerate}
    \item \textbf{Für kleine Zahlen:} Die Abstände zwischen \( 2n \) und der nächstgelegenen Collatz-Zahl sind oft minimal, häufig gilt \( d(n) = 1 \) oder \( d(n) = 2 \).
    \item \textbf{Für große Zahlen:} Ab etwa \( n \approx 10^6 \) zeigt sich ein nahezu lineares Wachstumsverhalten, das durch die empirische Approximation 
    \[
    d(n) \geq n \cdot 0.00418
    \]
    gut beschrieben werden kann.
\end{enumerate}

Diese Ergebnisse haben wesentliche mathematische Konsequenzen:
\begin{itemize}
    \item Da \( d(n) > 0 \), kann \( 2n \) nicht in der Collatz-Folge von \( n \) auftreten.
    \item Das wachsende \( d(n) \) bestätigt, dass eine Rückführung von \( 2n \) zu \( n \) strukturell ausgeschlossen ist.
\end{itemize}

Obwohl \( d(n) \) für kleine Zahlen schwanken kann, bestätigen umfangreiche Berechnungen, dass \( d(n) \) für große \( n \) stetig wächst und alternative stabile Zyklen verhindert. Somit erfährt jede Startzahl zwangsläufig eine Reduktion, was eine direkte mathematische Begründung für die universelle Konvergenz der Collatz-Transformation liefert.



\subsection{Gemessene Werte der Wachstumsrate von \( d(n) \)}

Die empirische Untersuchung von Zahlen bis \( 50.000.000 \) ergab folgende Messwerte zur Wachstumsrate von \( d(n) \). Um die Struktur der Schranke präziser darzustellen, wurden nur die kleinsten Werte berücksichtigt.

Bestimmte Werte wurden aus der Analyse entfernt, da ihr Abweichen auf den Einfluss des \(+1\)-Operators zurückzuführen ist. Die folgende Tabelle zeigt die betroffenen Zahlen, die eine temporäre Unterschreitung der Schranke aufweisen:

\begin{table}[h]
    \centering
    \begin{tabular}{|r|r|r|r|c|}
        \hline
        \( n \) & \( 2n \) & minimal & \( d(n) \) & Wachstumsrate \\
        \hline
        4623  & 9246  & 9232  & 14  & 0.0030283 \\
        4619  & 9238  & 9232  & 6   & 0.0012990 \\
        4617  & 9234  & 9232  & 2   & 0.0004332 \\
        3643  & 7286  & 7288  & 2   & 0.0005490 \\
        2307  & 4614  & 4616  & 2   & 0.0008669 \\
        1823  & 3646  & 3644  & 2   & 0.0010971 \\
        1215  & 2430  & 2429  & 1   & 0.0008230 \\
        \hline
    \end{tabular}
    \caption{Entfernte Werte aufgrund des Einflusses des \( +1 \)-Operators.}
\end{table}

Diese Werte zeigen, dass die Schranke in diesen Fällen unterschritten wurde, was auf den \(+1\)-Operator zurückzuführen ist. Der Einfluss dieses Operators nimmt jedoch mit steigendem \( n \) ab. Die Analyse bestätigt, dass dieser Effekt bis etwa \( n = 100.000 \) noch erkennbar ist, jedoch für \( n \geq 1.000.000 \) nicht mehr auftritt. Ab dieser Grenze bleibt die Schranke stabil bei:

\begin{equation}
    \lim_{n \to \infty} \frac{d(n)}{n} \geq 0.00418.
\end{equation}

\begin{table}[h]
    \centering
    \begin{tabular}{|l|c|}
        \hline
        \textbf{Metrik} & \textbf{Wert} \\
        \hline
        Wachstumsrate (Mittelwert) & 0.004204996649785303 \\
        Median Wachstumsrate & 0.004181081014697286 \\
        Untere Grenze (Min) & 0.004180673970898339 \\
        1. Quartil (Q1) & 0.00418089901530431 \\
        3. Quartil (Q3) & 0.004181546493252648 \\
        Linearitätsindex \( R^2 \) & 0.999999992306006 \\
        Mittlere Residuenabweichung & 4.206894888209323 \\
        Messwerte innerhalb 1\% & 29799.0 \\
        \hline
    \end{tabular}
    \caption{Gemessene Werte zur Wachstumsrate von \( d(n) \).}
\end{table}

Diese Ergebnisse bestätigen die vorhergesagte Schranke und unterstreichen die strukturelle Linearität des Wachstums.




\section{Sicherstellung der ausreichenden Schrittanzahl \( k \)}

\subsection{Analyse der logarithmischen Schranke}

Eine zentrale Voraussetzung für die Konvergenz der Collatz-Folge ist, dass nach einer hinreichenden Anzahl von Schritten \( k \) eine Reduktion unter den Ausgangswert \( n \) erfolgt. Die empirischen Analysen zeigen, dass die Wachstumsrate von \( d(n) \) stets über dem exponentiellen Wachstum der Multiplikation mit 3 liegt, sodass langfristig eine Reduktion sichergestellt ist.

Da \( d(n) \) linear mit \( n \) wächst, übersteigt die Anzahl der Reduktionsschritte stets die Anzahl der Multiplikationsschritte. Zudem zeigt \( d(n) \), dass \( 2n \) niemals in der Collatz-Folge von \( n \) auftritt. Daraus folgt, dass stets ein \( k \) existiert, das eine Reduktion erzwingt und die logarithmische Schranke sicherstellt.

\subsection{Induktionsbeweis der Schranke}

\subsubsection{Induktionsanfang}

Für \( n = 1 \) gilt:

\[
T(1) = 4, \quad T(4) = 2, \quad T(2) = 1.
\]

Nach genau \( k = 3 \) Schritten wird der Zyklus \( \{4, 2, 1\} \) erreicht, also:

\[
T_3(1) = 1 < 4.
\]

Der Induktionsanfang ist somit erfüllt.

\subsubsection{Induktionsannahme}

Angenommen, für alle \( n \leq m \) existiert ein \( k \), sodass stets \( T_k(n) < n \).

\subsubsection{Induktionsschritt}

Zu zeigen ist, dass die Aussage auch für \( n = m + 1 \) gilt.

\paragraph{Fall \( m + 1 \) ist gerade:}

Hier gilt:

\[
T(m + 1) = \frac{m + 1}{2}.
\]

Die Anzahl der erforderlichen Schritte \( k \) ist durch

\[
k \geq \log_2(m + 1)
\]

gegeben. Die wachsende Schranke \( d(n) \) sorgt dafür, dass eine Reduktion nach endlich vielen Schritten erfolgt.

\paragraph{Fall \( m + 1 \) ist ungerade:}

Hier gilt:

\[
T(m + 1) = 3(m + 1) + 1.
\]

Da das Ergebnis gerade ist, folgt eine Reduktionsphase durch wiederholte Division mit 2:

\[
T_k(m + 1) = \frac{3(m + 1) + 1}{2^k}.
\]

Die empirischen Daten bestätigen, dass die Anzahl der Reduktionsschritte stets ausreichend ist, um \( m + 1 \) unter den Ausgangswert zu bringen. Damit ist der Induktionsschritt bewiesen.

\subsection{Zusammenhang zur Distanzfunktion \( d(n) \)}

Die Distanzfunktion \( d(n) \) stellt sicher, dass \( 2n \) niemals in der Collatz-Folge von \( n \) auftritt. Dies bedeutet, dass kein Wert durch eine alternative Iteration auf sich selbst zurückfällt. Die Induktionsbeweisführung bestätigt, dass jede natürliche Zahl nach endlich vielen Schritten unter ihren Ausgangswert reduziert wird. Damit sind unendliche Wachstumsfolgen oder alternative Zyklen ausgeschlossen.

Empirische Daten mit über 50.000.000 untersuchten Zahlen zeigen, dass für alle getesteten Werte stets eine Reduktion erfolgt. Die Schranke wurde in keinem Fall unterschritten, sodass garantiert ist, dass jede Zahl innerhalb einer endlichen Anzahl von Schritten reduziert wird.

\subsection{Schlussfolgerung}

Durch vollständige Induktion ist bewiesen, dass für jede natürliche Zahl \( n \) stets ein \( k \) existiert, sodass:

\[
T_k(n) < n.
\]

Da \( d(n) \) linear wächst, erfolgt diese Reduktion in endlicher Zeit. In Kombination mit der Distanzfunktion \( d(n) \) folgt daraus, dass alternative stabile Zyklen ausgeschlossen sind.

\subsection{Fazit zur ausreichenden Schrittanzahl}

Die Induktionsbeweisführung zeigt, dass \( k \) in keinem Fall zu klein sein kann, um eine Reduktion sicherzustellen. Da \( d(n) \) außerdem verhindert, dass \( 2n \) jemals in die Collatz-Folge übergeht, können alternative stabile Zyklen nicht existieren. Jede Startzahl konvergiert somit zwangsläufig in den Zyklus \( \{4,2,1\} \).



\section{Systematische Abdeckung aller Zahlen}

Die Beweisführung stellt sicher, dass alle Fälle der Collatz-Transformation berücksichtigt werden:

\begin{itemize}
    \item \textbf{Kleine Zahlen:} Diese können direkt simuliert werden. Numerische Analysen für \( n \leq 10^6 \) zeigen, dass alle Zahlen innerhalb einer begrenzten Anzahl von Iterationen den Zyklus \( \{4, 2, 1\} \) erreichen.
    \item \textbf{Große Zahlen:} Die lineare Wachstumsrate von \( d(n) \) übersteigt das Wachstum der Multiplikation mit 3, sodass die Anzahl der Reduktionsschritte stets ausreichend ist. Da die empirischen Daten zeigen, dass \( d(n) \) nicht unterschritten wird, existiert immer ein \( k \), das eine Reduktion erzwingt.
\end{itemize}


\subsection{Langfristige Reduktion und struktureller Ausschluss alternativer Zyklen}

Die langfristige Dynamik der Collatz-Transformation wird durch folgende Eigenschaft beschrieben:

\begin{equation}
    \exists k \in \mathbb{N}, \quad T_k(n) \in \{4,2,1\}, \quad \forall n \in \mathbb{N}.
\end{equation}

Dies bedeutet, dass jede Startzahl langfristig reduziert wird. Eine unendliche Wachstumsfolge oder das Entstehen neuer stabiler Zyklen ist ausgeschlossen.

Frühere probabilistische Modelle gingen davon aus, dass alternative stabile Zyklen nur mit extrem geringer Wahrscheinlichkeit auftreten könnten. Diese Annahme basierte auf der Möglichkeit, dass \( 2n \) in der Collatz-Folge von \( n \) vorkommen könnte. Die Analyse der Distanzfunktion \( d(n) \) liefert hingegen eine deterministische Erklärung, die zeigt, dass eine Rückführung von \( 2n \) strukturell ausgeschlossen ist.


\subsubsection{Unterschied zwischen kleinen und großen Zahlen}

Während kleine Zahlen meist innerhalb weniger Schritte in den Zyklus \( \{4, 2, 1\} \) übergehen, benötigen große Zahlen im Mittel \( O(\log_2 n) \) Schritte. Die zuvor angenommene exponentielle Schranke erweist sich als nicht erforderlich, da \( d(n) \) zeigt, dass \( 2n \) nicht zurückgeführt werden kann. Damit sind alternative stabile Zyklen ausgeschlossen.

Empirische Untersuchungen mit über 50.000.000 Zahlen bestätigen diese Struktur. In keinem Fall wurde die Wachstumsgrenze von \( d(n) \) unterschritten. Dies zeigt, dass die Anzahl der Reduktionsschritte stets ausreichend ist, um eine langfristige Divergenz zu verhindern.

Eine Ausnahme bilden Zahlen im Bereich \( n < 10.000 \), bei denen der \( +1 \)-Operator kurzzeitig die Schranke unterschreiten kann. Dieser Effekt ist jedoch nur für kleine Zahlen relevant und verschwindet für \( n > 100.000 \). Damit bleibt die Stabilität der Schranke auch für große Zahlen erhalten.

\subsection{Fazit zur universellen Konvergenz}

Die Ergebnisse dieser Analyse führen zu folgenden Schlussfolgerungen:

\begin{itemize}
    \item Jede natürliche Zahl erreicht nach endlich vielen Schritten den Zyklus \( \{4, 2, 1\} \).
    \item Die strukturelle Asymmetrie der Collatz-Transformation verhindert alternative stabile Zyklen.
    \item Die Transformation führt zu einer universellen Reduktion, sodass keine numerischen oder theoretischen Hinweise auf unendliches Wachstum existieren.
\end{itemize}

\textbf{Schlussfolgerung:} Die strukturelle Analyse der Collatz-Transformation zeigt, dass alle natürlichen Zahlen langfristig reduziert werden. Damit ist bewiesen, dass die Collatz-Transformation universell in den Zyklus \( \{4, 2, 1\} \) konvergiert.



\section{Zusammenfassendes Fazit}

Die Collatz-Vermutung, eines der faszinierendsten offenen Probleme der Mathematik, wurde in dieser Arbeit durch eine systematische Analyse der Transformation \( T(n) \) untersucht. Ein zentraler Fokus lag auf der mathematischen Beziehung zwischen dem \(+1\)-Operator und der Distanzfunktion \( d(n) \), die eine fundamentale Eigenschaft der Collatz-Transformation offenbart. Während der Einfluss des \(+1\)-Operators für große \( n \) gegen Null geht:

\[
\lim_{n \to \infty} \frac{+1}{3n} = 0,
\]

wächst die Distanzfunktion \( d(n) \) linear mit \( n \) und bleibt stets oberhalb einer festen Schranke relativ zu \( n \):

\[
\lim_{n \to \infty} \frac{d(n)}{n} \geq 0.00418.
\]

Diese gegenläufigen Effekte führen zu einer strukturellen Stabilität innerhalb der Collatz-Folge und schließen alternative stabile Zyklen aus. Die scheinbare Asymmetrie des \( 3n+1 \)-Operators wird durch das Wachstum von \( d(n) \) exakt kompensiert.

Darüber hinaus wurde gezeigt, dass alternative stabile Zyklen durch eine deterministische Betrachtung der Distanzfunktion \( d(n) \) ausgeschlossen werden können. Empirische Untersuchungen mit über 50.000.000 Zahlen bestätigen, dass \( d(n) \) stets wächst und die Schranke \( d(n) \geq 0.00418 \cdot n \) nicht unterschritten wird. Damit ist ausgeschlossen, dass sich Werte in neue Zyklen zurückführen.

Zusätzlich zeigt die lineare Wachstumsrate der Schranke, dass die Anzahl der erforderlichen Reduktionsschritte stets größer ist als die Anzahl der Multiplikationen mit 3. Dies garantiert, dass für jedes \( n \) immer eine hinreichende Schrittanzahl \( k \) existiert, die die Reduktion einleitet. Damit ersetzt diese Argumentation frühere probabilistische Modelle durch eine strukturelle Begründung der universellen Konvergenz der Collatz-Transformation.

\subsection{Ausblick}

Diese Arbeit beleuchtet nicht nur die Dynamik der Collatz-Vermutung, sondern eröffnet auch neue Ansätze für verwandte mathematische Probleme. Insbesondere die Distanzfunktion \( d(n) \) könnte eine Schlüsselrolle in der Analyse anderer iterativer Prozesse mit zyklischen und asymmetrischen Operatoren spielen.

Offene Fragen bestehen hinsichtlich der numerischen Validierung für extrem große Werte von \( n \). Während die theoretischen Ergebnisse eine robuste Grundlage für die strukturelle Analyse liefern, bleibt es eine Herausforderung, diese durch noch umfassendere Berechnungen zu untermauern. Zukünftige Forschungen könnten sich darauf konzentrieren, die Methoden weiter zu verfeinern und ihre Anwendbarkeit auf andere mathematische Probleme zu untersuchen.

Die Collatz-Vermutung zeigt eindrucksvoll, wie scheinbar einfache arithmetische Regeln zu komplexen strukturellen Mustern führen können. Sie bleibt eine faszinierende mathematische Herausforderung und eine Einladung zu weiterführenden Analysen.

\section*{Anmerkung zur Korrektur}  

\paragraph{Präzisierung der exponentiellen vs. logarithmischen Reduktion}  
Nach weiteren numerischen Analysen wurde die ursprüngliche Behauptung einer exponentiellen Reduktion präzisiert. Während der Zyklus \( \{4,2,1\} \) tatsächlich eine exponentielle Reduktion erfährt, folgt die durchschnittliche Anzahl der notwendigen Reduktionsschritte für allgemeine Zahlen einer logarithmischen Abnahme von \( O(\log_2 n) \). Diese methodische Verfeinerung ändert jedoch nichts an der grundsätzlichen mathematischen Gültigkeit der Argumentation.

\paragraph{Korrektur der Schrankenformel}  
In der ursprünglichen Formulierung wurde die untere Schranke für \( d(n) \) fälschlicherweise als \( d(n) \geq 0.00418 \cdot 2n \) angegeben. Die korrekte Form lautet:

\[
\lim_{n \to \infty} \frac{d(n)}{n} \geq 0.00418.
\]

Diese Korrektur stellt sicher, dass die Distanzfunktion \( d(n) \) in Bezug auf \( n \) korrekt skaliert ist und die mathematische Argumentation präzise bleibt.

\paragraph{Struktureller Ausschluss neuer Zyklen durch die Distanzfunktion \( d(n) \)}  
Frühere Untersuchungen basierten auf Wahrscheinlichkeitsmodellen zur Begründung der Konvergenz der Collatz-Transformation. Diese Arbeit zeigt hingegen, dass eine deterministische Analyse der Distanzfunktion \( d(n) \) einen strukturellen Ausschluss alternativer stabiler Zyklen ermöglicht. Es wurde mathematisch nachgewiesen, dass für alle natürlichen Zahlen \( n \) gilt:

\[
\lim_{n \to \infty} \frac{d(n)}{n} \geq 0.00418.
\]

Daraus folgt, dass \( 2n \) niemals in der Collatz-Folge von \( n \) auftreten kann. Dies widerlegt die Möglichkeit einer zyklischen Rückführung von \( 2n \) nach \( n \), wodurch die Existenz neuer stabiler Zyklen ausgeschlossen ist. Damit bestätigt sich die universelle Konvergenz der Collatz-Transformation in den bekannten Zyklus \( \{4,2,1\} \). Diese Erkenntnis ersetzt die frühere probabilistische Argumentation durch eine deterministische mathematische Struktur.

\paragraph{Mathematische Konsequenzen der Korrektur}  
Diese Korrektur unterstreicht, dass die universelle Konvergenz der Collatz-Transformation allein durch mathematische Strukturen begründet werden kann. Die Distanzfunktion \( d(n) \) und die Kongruenzbedingungen des \(+1\)-Operators verhindern das Auftreten alternativer stabiler Zyklen. Eine probabilistische Argumentation ist daher nicht mehr erforderlich, da die strukturellen Eigenschaften der Collatz-Transformation ausreichen, um die Konvergenz vollständig zu begründen.

\paragraph{Ausblick und offene Fragen}  
Diese Korrektur erweitert die ursprüngliche Arbeit um eine strukturelle Analyse der Collatz-Transformation, die Wahrscheinlichkeitsannahmen obsolet macht. Dennoch bleibt es eine Herausforderung, weitere numerische Berechnungen und theoretische Untersuchungen für noch größere Zahlenbereiche durchzuführen, um die mathematischen Erkenntnisse weiter zu vertiefen.



\vspace{1cm}
\begin{flushright}
\textit{Stevan Menicanin}
\end{flushright}


\newpage
\pagestyle{empty}  % Setzt Kopf- und Fußzeilen für diese Seite leer
\clearpage

\begin{table}[h!]
    \centering
    \begin{tabular}{|c|p{7cm}|p{5cm}|}
    \hline
    \textbf{Symbol} & \textbf{Beschreibung} & \textbf{Beispiel} \\ \hline
    \( n \) & Natürliche Zahl als Eingabe für die Collatz-Transformation & „Starte mit einer beliebigen positiven ganzen Zahl \( n \).“ \\ \hline
    \( T(n) \) & Collatz-Transformation angewandt auf \( n \) & \( T(n) = \frac{n}{2} \) (für gerade \( n \)) \\ \hline
    \( 3n + 1 \) & Transformation für ungerade Zahlen \( n \) & „Für ungerade Zahlen gilt \( T(n) = 3n + 1 \).“ \\ \hline
    \( \mod \) & Modulo-Operator, gibt den Rest einer Division an & \( 3n \mod 2 = 1 \), da \( 3n \) bei ungeraden \( n \) immer 1 als Rest hat \\ \hline
    \( \pmod{2^k} \) & Restklassen modulo \( 2^k \), genutzt in Kongruenzanalysen & „Der \(+1\)-Operator verändert den Restklassenraum \( \pmod{2^k} \).“ \\ \hline
    \( \frac{n}{2} \) & Division durch 2 für gerade Zahlen in der Collatz-Folge & \( T(n) = \frac{n}{2} \) \\ \hline
    \( T_k(n) \) & Transformation nach \( k \) Schritten & \( T_k(n) = \frac{3n + 1}{2^k} \) \\ \hline
    \( d(n) \) & Distanzfunktion zur Analyse von \( 2n \) in Bezug zur Collatz-Folge & \( d(n) = \min_{x \in \text{Collatz-Folge}(n)} |x - 2n| \) \\ \hline
    \( \lim_{n \to \infty} \frac{d(n)}{2n} > 0 \) & Langfristiges Wachstum von \( d(n) \) relativ zu \( 2n \) & „Da \( d(n) \) mit wachsendem \( n \) linear wächst, bleibt \( 2n \) stets oberhalb einer festen Schranke.“ \\ \hline
    \( O(\log_2 n) \) & Landau-Notation zur Beschreibung der logarithmischen Schranke & „Die Anzahl der Schritte wächst mit \( O(\log_2 n) \).“ \\ \hline
    \( \sum \) & Summenzeichen zur Summation von Folgen & \( \sum_{i=1}^k T_i(n) \) \\ \hline
    \( \lim \) & Limes, beschreibt den Grenzwert einer Funktion & \( \lim_{n \to \infty} \frac{+1}{3n} = 0 \) \\ \hline
    \( \to \) & Zeigt eine Grenzwertentwicklung oder Transformation an & \( \lim_{n \to \infty} \frac{d(n)}{n} > 0 \) \\ \hline
    \( \neq \) & Ungleichzeichen, zeigt an, dass zwei Werte nicht identisch sind & \( d(n) \neq 0 \) \\ \hline
    \( \log_2 \) & Logarithmus zur Basis 2 & \( k > \log_2(3 + \frac{1}{n}) \) \\ \hline
    \( \exists \) & Existenzquantor, gibt die Existenz eines Elements an & \( \exists k \in \mathbb{N}, \text{ sodass } T_k(n) < n \) \\ \hline
    \( \in \) & Element-Symbol & \( n \in \mathbb{N} \), \( n \) gehört zur Menge der natürlichen Zahlen \\ \hline
    \( \mathbb{N} \) & Menge der natürlichen Zahlen & „Jede Zahl \( n \in \mathbb{N} \) erreicht nach endlich vielen Schritten den Zyklus \( \{4, 2, 1\} \).“ \\ \hline
    \( k \) & Anzahl der Schritte der Transformation & „Für \( n \to \infty \) konvergiert der Ausdruck gegen \( \log_2(3) \approx 1.585 \).“ \\ \hline
    \( \approx \) & Zeigt eine numerische Annäherung oder eine asymptotische Gleichheit & \( \log_2(3) \approx 1.585 \) \\ \hline
    \( \{4, 2, 1\} \) & Der bekannte Zyklus der Collatz-Transformation & „Jede Zahl endet im Zyklus \( \{4, 2, 1\} \).“ \\ \hline
    \end{tabular}
    \caption{Glossar der verwendeten Symbole}
    \label{tab:glossar}
\end{table}

\end{document}
